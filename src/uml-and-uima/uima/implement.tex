
\section{Adding descriptors and codes for your name entity recognizer}

If you have one or several idea(s) for the Gene Mention tagging task already,
it's time to implement all of them in this task. You will find the official
tutorial is helpful for you to accomplish this task. Except that you are
required to build your component based on UIMA framework, there is little
limitation on how you should design and implement your type system as well as
all the pipeline components. Nevertheless, there are still some suggestions (or
you can say tips or our expectations) to consider.

\subsection{Type system}

\begin{enumerate}

\item The minimal type system to accomplish this task should include types for
input and output elements, i.e. the id and text of each sentence and gene tags.

\item If you intend to design a pipeline of two or more phases (e.g., each of
the first few phases extracts a certain feature, and the last phase predicts the
gene mentions based on all the features), you should include all the
intermediate types in your type system as well.

\item If your type system is complex, then you can try to refactor the type
system by categorizing the types according to their functions (e.g., NLP
related, biological related, etc.), and creating multiple type systems for each
of the categories.

\item Another common practise in the type system design is to create a
\emph{base annotation} type which includes two features: a string feature
\emph{source} and a numeric feature \emph{confidence} to help keep track of
where an annotation was originally made by, and how confidence the annotation
was. All other types then inherit from this base annotation type.

\item You could assign a (or several) name space(s) for your types, e.g., instead
of \texttt{Sentence} (default name space), you could name it
\texttt{model.Sentence} (a type with name space \texttt{model}). Once you click
the \textbf{JCasGen} button, you will find a nice hierarchical folder structure
created under \texttt{src/main/java} for the Java classes corresponding to the
types.

\end{enumerate}

\subsection{Collection processing engine}

\begin{enumerate}

\item You can use UML to help you and us clarify your pipeline design. You could
include some UML diagrams in the Javadocs and in your report.

\item You can refer to the \texttt{SimpeRunCPE.java} in the uima-example package
(and we also included it in the archetype) as the entry point to test your
pipeline once you have done all the implementations.

\item The UIMA Eclipse plug-in has not provided a GUI for you to specify the CPE
descriptor yet, which means although it will not be complicated, you still need to
manually create the CPE descriptor all by yourself.

\item \textbf{Please name your CPE descriptor as CpeDescriptor.xml and put it
under \texttt{src/main/resources/}}, so that we could easily find the entry point
of your pipeline and locate all your component descriptors and thus your
component implementations.

\end{enumerate}

\subsection{Collection reader}

\begin{enumerate}

\item Since your pipeline is required to take an arbitrary file (with the same
format as \texttt{sample.in}) as input, you should define a configuration
parameter for the input file path in the collection reader. It could be
``src/main/resources/data/sample.in'' (without quotes) for your own test, but
\textbf{in your final submission, please put ``hw1.in'' (without quotes) as the
value for the parameter}, which is the path of the file used to test your
pipeline.

\item Ideally, depending on where the input file is located (on a local file
system, in a jar file, or from an external sources, e.g. network), your
collection reader needs to be general enough to establish a connection to the
file source and open a stream to read the content. But for our task, you only
need to consider a file located on the file system, which means the basic
\texttt{FileInputStream} and \texttt{FileReader} will do the work.

\end{enumerate}

\subsection{Cas consumer}

\begin{enumerate}
  
\item Please refer to a prior section to find out the expected output format.

\item Similar to the collection reader, you should define a configuration
parameter for the output file path. \textbf{Please name it as ``hw1-ID.out''
(without quotes) in your final submission, where ID is again your Andrew ID},
and then your cas consumer will write the output file to the project root
directly.

\item You could create multiple views for a CAS, e.g., one for all the input
elements, and one for all the intermediate results, and one for the final
annotations, then your cas consumer can read the document ID from the first
view, and the annotations from the last view to produce the output file. If you
don't do it in this way, then you should probably include the document ID as a
feature in all the intermediate types and final annotation type.

\end{enumerate}

\subsection{Annotator}

\begin{enumerate}

\item If you want to employ a resource (e.g., a model you trained offline) in
your annotator, you could consider UIMA's \emph{resource manager} (refer to the
official tutorial for details about this).
Be sure to put your resource in \texttt{src/main/resources} so that your
submission will also bundle those resources along with your codes.

\item \textbf{If you want to incorporate other NLP or machine learning tools,
please try to avoid non-Java packages. If you want to dependent your artifact on
other Java packages other than those provided by the archetype but can be found
in the course repository, you can add a Maven dependency for this artifact, if
you have the jar package on your machine, but it doesn't exist in the course
repository, please let us know, we will try to deploy them in a 3rd party
repository.}

\item Remember to add comments and Javadocs to your annotators, we will also
evaluate the quality of your codes.

\item The most creative ideas will happen in the intermediate annotators. We
have mentioned some possible solutions to do this. You can try and implement
multiple approaches, and annotate the gene mentions for the sentence multiple
times. All the annotations should be kept in the CAS until a final ``merging''
component takes all the annotations and make a final judgment.

You can use the type's \emph{source} feature to identify, among all the
approaches, which annotator is attributed this annotation, and \emph{confidence}
feature as a weight to vote for a final decision. This approach was also applied
in IBM's Watson system.

\end{enumerate}
