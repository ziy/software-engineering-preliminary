% 
% This work is licensed under the Creative Commons Attribution-ShareAlike 3.0
% Unported License. To view a copy of this license, visit
% http://creativecommons.org/licenses/by-sa/3.0/.
%
% author: Zi Yang <ziy@cs.cmu.edu>
% date: 09/02/2012
%
\documentclass[oneside]{memoir}

\renewcommand{\chaptername}{Task}

\usepackage{times}

\usepackage{titlesec}
\titleformat{\section}{\normalfont\Large\bfseries}{Task \thesection}{1em}{}

\usepackage{url}
\usepackage{hyperref}

\usepackage{graphicx}
\graphicspath{{../../fig/cse-and-qa/}}

\usepackage{pstricks}
\usepackage{epsfig}

\usepackage{listings}
\usepackage{color}
\definecolor{dkgreen}{rgb}{0,0.6,0}
\definecolor{gray}{rgb}{0.5,0.5,0.5}
\definecolor{lightblue}{rgb}{0.95,0.95,1}
\definecolor{mauve}{rgb}{0.58,0,0.82}
\lstset{
  basicstyle=\ttfamily\small,
  numbers=left,
  numberstyle=\tiny\color{gray},
  stepnumber=2, 
  numbersep=5pt,
  backgroundcolor=\color{lightblue},
  showspaces=false,
  showstringspaces=false,
  showtabs=false,
  frame=lines,
  rulecolor=\color{black},
  tabsize=2,
  captionpos=b,
  breaklines=true,
  breakatwhitespace=false,
  title=\lstname,
  keywordstyle=\color{blue},
  commentstyle=\color{dkgreen},
  stringstyle=\color{mauve},
  escapeinside={\%*}{*)},
  morekeywords={*,...},
}
\usepackage{letltxmacro}
\makeatletter
\LetLtxMacro\@@lst@inputlisting\lst@inputlisting
\renewcommand\lst@inputlisting[2][]{%
  \try@listingspath{#2}%
  \if@tempswa
    \typeout{Using \@foundlisting}%
    \@@lst@inputlisting[#1]{\@foundlisting}%
  \else
    \typeout{Missing file #2}\endgroup
  \fi}
\def\listingspath#1{\def\@listingspath{#1}}
\listingspath{}
\def\try@listingspath#1{%
  \@tempswafalse
  \expandafter\@tfor\expandafter\next
  \expandafter:\expandafter=\@listingspath\do
  {\if@tempswa\@break@tfor\fi
   \IfFileExists{\next/#1}{\@tempswatrue\xdef\@foundlisting{\next/#1}}{}}%
}
\makeatletter
\listingspath{{../../lst/cse-and-qa/}}

\usepackage{multirow}

\definecolor{shadecolor}{gray}{0.9}

\newenvironment{qa}
{\begin{shaded}\begin{itemize}}
{\end{itemize}\end{shaded}}

\title{{\bfseries 11-791 Design and Engineering of Intelligent Information
System Fall 2012\\Homework 2}\\
\vspace{1em}
\itshape\rmfamily Configuration Space Exploration for Biomedical Question
Answering}

\date{}

\begin{document}

\begin{titlingpage}
\maketitle

\hspace{-0.1\textwidth}
\begin{minipage}{1.2\textwidth}
\vspace{-5em}
\textbf{Important dates}
\begin{itemize}

\item \textbf{Hand out: November 1.}\footnote{This version was built on \today}

\item \textbf{Milestone 1 (M1) turn in: November 8.}

\begin{itemize}

\item You are required to submit all your Java source codes and YAML descriptors
for your components, and you DON'T need to submit any other documents. Your main
YAML descriptor should include a single phase of multiple options for NERs.

\item In addition, please send us the URL of your project repository page (e.g.,
\url{https://github.com/ID/hw2-teamXX}). We will look into your Issues page and
Wiki page, and expect you have created milestones and issues in the Issues
system, and reported the results of your named entity recognizers evaluated on
your own, your team meeting minutes and probably your project goals or other any
important things in your Wiki page.

\end{itemize}

\item \textbf{Milestone 2 (M2) turn in: November 22.}

\begin{itemize}

\item You are required to submit all your Java source codes and YAML descriptors
for your components, and you DON'T need to submit any other documents. You main
YAML should include a complete pipeline, where each phase can only have a SINGLE
component. You can pick the best component for each phase based on your own
cross-option evaluation with CSE.

\item We will again look into your Issues page and Wiki page, and expect you
properly solved your previous created milestones and issues, and probably
created new milestones and issues, and reported the evaluation results of your
M2 on your own, your team meeting minutes and probably your revised project
goals.

\end{itemize}

\end{itemize}

\end{minipage}
\hspace{-0.1\textwidth}

\hspace{-0.1\textwidth}
\begin{minipage}{1.2\textwidth}

\begin{itemize}

\item \textbf{Milestone 3 (M3) turn in: December 6.}

\begin{itemize}

\item You are required to submit all your Java source codes and YAML descriptors
for your components. You main YAML should include a complete pipeline, where
each phase can only have a SINGLE component. You can pick the best component for
each phase based on your own cross-option evaluation with CSE.

\item Name your presentation slides as
\texttt{hw2-teamXX.ppt}\footnote{Or other formats, e.g., pptx, odp, pdf, etc.}
and put it in the \texttt{src/main/resources/docs}.

\item We expect you properly solved all the issues and accomplish all
milestones, and reported the evaluation results of your final system, your team
meeting minutes and a final summary.

\end{itemize}

\end{itemize}

\end{minipage}
\hspace{-0.1\textwidth}

\hspace{-0.1\textwidth}
\begin{minipage}{1.2\textwidth}

You should always organize your project in the same hierarchy as shown
below for all your submissions:

\small
\begin{verbatim}
hw2-teamXX
|- pom.xml
|- launches
|  `- hellobioqa.launch
|- data                           /* git-ignore this folder, since
|  `- oaqa-eval.db3                * it will become huge */
`- src
     `- main
        |- java/edu/cmu/lti/oaqa/openqa/test/teamXX
      |   `- **/*.java             /* your Java codes go into this 
      |                            * folder as you did before */
        `- resources
           |- input
           |- gs
           `- hellobioqa
              |- collection        /* descriptors for collection reader,
              |- keyterm            * example keyterm extractors,
              |- retrieval          * example retrieval strategists,
              |- passage            * example passage extractors */
              |- hellobioqa.yaml   /* the entry point for the example 
              |                     * pipeline and your pipeline */
              `- teamXX            /* your descriptors go into here */
                 `- **/*.yaml

\end{verbatim}
\normalsize

Several notes about organizing your Maven project and other additional
information:

\begin{enumerate}

\item \textbf{Submission:} The same way as you did for Homework 0 and 1 (set up
GitHub repo, create Maven project, write your code, submit to Maven repo),
except that the name has changed to hw2-teamXX (XX is a two-digit number from 01
to 18).

\item \textbf{Your source files and descriptors:}
\texttt{java/edu/cmu/lti/oaqa/openqa/test/teamXX} means you should create
directories (or packages in terms of Java program structure) iteratively from
\texttt{java}, all the way to \texttt{teamXX}.

\verb|**/*.java| and \verb|**/*.yaml| tell you that you don't need to flatten
your folder hierarchy, instead we encourage you to place your Java codes in the
right package, and similarly, you can create subfolders for different types of
descriptors, e.g., \verb|src/main/resources/hellobioqa/teamXX/keyterm| for all
the descriptors related to keyterm extraction task.

Actually, once you specify the main yaml descriptor of pipeline to the
\texttt{ECDDriver}, you can have your own entry point to the system. However, if
you want to use \texttt{launches/hellobioqa.launch} file to run the pipeline,
you should consider \texttt{src/main/resources/hellobioqa/hellobioqa.yaml} as
your main yaml. When we evaluate your codes, we will run a different main yaml
to bundle all your components, all the components declared in your
\texttt{hellobioqa.yaml} will be used.

\item \textbf{Comments, Javadocs, and documentations:} They are still important
if you want us and other users to better understand your code. (Remember: we
will become your customer when we run the big experiment.)

\end{enumerate}

\end{minipage}
\hspace{-0.1\textwidth}

\hspace{-0.1\textwidth}
\begin{minipage}{1.2\textwidth}

\textbf{Useful information}
\begin{enumerate}

\item Please visit the course blackboard regularly to check if a newer version
is published. We may have new versions or revised instruction at the begining of
each milestone. Since this is still an ongoing project, we might also fix bugs
or enhance features to the framework. If an urgent bug fix is done, we will also
make an announcement on the blackboard.

\item If you have difficulties in using Eclipse, Java, git, GitHub, Maven,
Sonatype Nexus, Lucene, Solr, etc., you will find Google is the still the most
effective way to solve any problem. But if you find the problem might exist in
the framework, the quickest solution is to post an issue on the GitHub ISSUES
system, and all the developers will be notified for the newly posted issue.

Please take a look at the following table to find out where to post an issue.

\vspace{1em}

\begin{tabular}{p{14em}lp{15em}}
\hline
Key words of your problem & Suspected project & Report it at \\
\hline
ECD, descriptor, phase, yaml, driver & uima-ecd &
\url{https://github.com/oaqa/uima-ecd/issues} \\
CSE, persistence-provider, retrieval evaluation & cse-framework &
\url{https://github.com/oaqa/cse-framework/issues} \\
QA, abstract classes, OAQA type system, keyterm evaluation, passage evaluation,
collection reader & baseqa &
\url{https://github.com/oaqa/baseqa/issues} \\
oaqa-eval, cse repository, database & jdbc-provider &
\url{https://github.com/oaqa/jdbc-provider/issues} \\
solr, retrieval, lucene & solr-provider &
\url{https://github.com/oaqa/solr-provider/issues} \\
example implementations & helloqa &
\url{https://github.com/oaqa/helloqa/issues} \\
trec-genomics adaptation & hellobioqa &
\url{https://github.com/ziy/hellobioqa/issues} \\
\hline
\end{tabular}

\vspace{1em}

For any question regarding the policy of Homework 2, please send
mails directly to Prof. Eric Nyberg
(\href{mailto:ehn@cs.cmu.edu}{\nolinkurl{ehn@cs.cmu.edu}}), and for other
questions about the homework, please send us mails: Zi Yang
(\href{mailto:ziy@cs.cmu.edu}{\nolinkurl{ziy@cs.cmu.edu}}) or Rui Liu
(\href{mailto:ruil@cs.cmu.edu}{\nolinkurl{ruil@cs.cmu.edu}}).

\item Again, both source files and pdf file of this assignment are
publicly available on my GitHub

\url{http://github.com/ziy/software-engineering-preliminary}

Please feel free to fork the project and send a pull request back to me as some
of you did for Homework 0 for any error. Or you can just report an issue at

\url{http://github.com/ziy/software-engineering-preliminary/issues}

\end{enumerate}

\end{minipage}
\hspace{-0.1\textwidth}

\end{titlingpage}


\chapter{Installing and Configuring Softwares}

In this task, you are required to install the tools you've already been familiar
with (at least heard about).

\textbf{Important notes:} This might be the hardest part of your Homework 0,
since installation of the same software on different platforms differs a lot,
which means we could not show you detailed steps for each platform. Instead, we
will provide you with links for the installation instructions for you to follow,
and we have created three forums for you to discuss Windows/Linux/Mac platform
related issues. We will try our best to help you solve whatever problems you
might have, but we also encourage you, the experts in any of these platforms, to
work with us to answer the questions from your classmates. You can find the
discussion boards at \url{http://blackboard.andrew.cmu.edu}, and select the
course, then the \textbf{Tools} and \textbf{Discussion Board}.


\section{Installing JDK}

If you have the latest JDK 6 installed\footnote{You should check out the latest
version at
\url{http://www.oracle.com/technetwork/java/javase/downloads/index.html} as the
version number grows really fast}, you could skip this task.

We assume you have experience in Java programming, but we still need to clarify
the Java environment for the course. If you don't have any Java experience,
probably you need to look for a Java textbook. It might also be fine if you
think you have tons of experience in programming in C++/C\# and you feel
confident to learn Java by just reading others' codes and guessing their
meanings. It's up to you!

\begin{enumerate}

\item Visit
\url{http://www.oracle.com/technetwork/java/javase/downloads/index.html}, and
choose the platform you are using to download JDK 6 SE 35\footnote{By the date
of August 31, 2012}.

\begin{qa}

\item[Q1] Can I just install JRE instead of JDK?

\item[A1] No.

\item[Q2] Can I install OpenJDK instead of SunJDK (or OracleJDK)?

\item[A2] Sure, you can. But be aware that sometimes only binary files (aka JRE)
are installed under a folder named \texttt{openjdk-\emph{version}}, rather than
\texttt{openjre-\emph{version}}, which is a bit confusing.

\item[Q3] Can I install JDK 7, 5 or older versions?

\item[A3] You are not recommended to install JDK 7, since you have to modify the
Maven pom file to compile your project, and the cluster that we will run and
test your components does not have JDK 7 set up yet. But it would be fine
(theoretically) if you have just JDK 5 installed, but it is still not
recommended. Versions older than 5 should be completely avoided.

\item[Q4] Can I use an earlier version of JDK 6 (e.g., 6u4)?

\item[A4] It may not put you in a trouble most of time, but in some rare cases,
we did find an exception was thrown due to a bug not in our code but in the
runtime environment. Therefore, we recommend you to upgrade your JDK 6 to the
latest version.

\end{qa}

\item Install JDK from the executable file if available, and set PATH manually
(if you are using a Windows machine). The Java installation page (at
\url{http://www.oracle.com/technetwork/java/javase/index-137561.html}) might be
useful to you.

\end{enumerate}


\section{Commiting and Pushing Your Maven Project Back to Git}

\begin{figure}[t]
\centering
\includegraphics[scale=0.3]{project-19-git-commit}
\caption{Performing a git-commit\label{project-19-git-commit}}
\end{figure}

You should see a ``greater than'' symbol between the project icon and your
project name (see Figure \ref{project-19-git-commit}, which means you have made
some changes to the project so that there exist some differences between your
current workspace version and the branch head. Question marks on the project
element icon represent the corresponding elements are not indexed yet. You might
be wondering what changes you've made because you thought you haven't written
any code. Actually, you have created a Maven project, which process will
generate the \verb|pom.xml| file, and modified the Eclipse configuration files.
So let's perform our first git-commit.

\begin{enumerate}

\item Right-click on the project name, and select \textbf{Team} $\rightarrow$
\textbf{Commit\ldots}. See Figure \ref{project-19-git-commit}, and then type in
your username and email address on GitHub in the popup ``Identify Yourself''
message window.

\begin{figure}[t]
\centering
\includegraphics[scale=0.3]{project-21-git-commit-message}
\caption{Viewing and confirming commit message\label{project-21-git-commit-message}}
\end{figure}

\begin{figure}[t]
\centering
\includegraphics[scale=0.3]{project-22-git-commit-done}
\caption{Committed project\label{project-22-git-commit-done}}
\end{figure}

\item In the ``Commit Changes'' window, you are allowed to choose the files you
want to commit (and also automatically add to the index). As you can see in
Figure \ref{project-21-git-commit-message}, during creating the empty Maven
project, several Eclipse and Maven related configuration files are generated.
Moreover, type in a commit message to describe what changes you've made and
leave your name as well as your email address (which is a convention) in the
committer field. Finally, by clicking \textbf{Commit}, you've done with your
first git-commit. You can see in Figure \ref{project-22-git-commit-done}, the
``greater than'' symbol disappears, and the question marks on the committed
files become a repository icon, which means the files are in the latest version.
You could also find your git-commit helps the code merge into the new master
branch from a NO-HEAD branch.

If you are using SVN, then you've done with sychronizing your local workspace
with the remote repository once you execute a commit command, but remember the
feature of Git? Your project repository is distributed, which means your
previous git-commit conceptually affects all your project repositories, but in
fact you should make it happen with an additional \emph{push}. A nice picture at
\url{http://gitready.com/beginner/2009/01/21/pushing-and-pulling.html} may help
you better understand what is actually happening when you perform git commit,
add, push, fetch, pull, etc.

\begin{figure}[t]
\centering
\includegraphics[scale=0.3]{project-23-git-push}
\caption{Performing a git push\label{project-23-git-push}}
\end{figure}

\item Right-click on the project name, and select \textbf{Team} $\rightarrow$
\textbf{Push to Upstream}. See Figure \ref{project-23-git-push}.

\begin{figure}[t]
\hspace{-2em}
\begin{minipage}{0.5\textwidth}
\centering
\includegraphics[scale=0.3]{project-24-git-push-progress}
\caption{Viewing the git push progress\label{project-24-git-push-progress}}
\end{minipage}
\hfill
\begin{minipage}{0.5\textwidth}
\centering
\includegraphics[scale=0.3]{project-25-git-push-result}
\caption{Viewing the git push result\label{project-25-git-push-result}}
\end{minipage}
\hspace{-1em}
\end{figure}

\item Now you can see a progress indication window pops up (see Figure
\ref{project-24-git-push-progress}), which says \textbf{Pushing to remote
repositories}.

\item Finally, you could see the push results in the ``Push Results'' window.
Click \textbf{OK} to close the window and get back to your workspace.

\end{enumerate}

In your next homework, you will need to use other Git commands within Eclipse.



\chapter{Release to Maven}

In the last task, you will need to submit your Java code by performing a release of your code. But remember that Maven release plug-in will check if all the changes you have made have been checked into the remote repository (i.e. GitHub in our case). So, let's perform a git-commit and a git-push.

\begin{enumerate}

\begin{figure}
\hspace{-1em}
\begin{minipage}{0.5\textwidth}
\centering
\includegraphics[scale=0.3]{simple-code-05-commit}
\caption{Performing a git-commit/push before preparing a release\label{simple-code-05-commit}}
\end{minipage}
\hfill
\begin{minipage}{0.5\textwidth}
\centering
\includegraphics[scale=0.3]{submit-01-preference}
\caption{Starting to add an external Maven executable\label{submit-01-preference}}
\end{minipage}
\hspace{-1em}
\end{figure}

\item Similar to what you did earlier, you execute git-commit and git-push to the project, and you could see the ``greater than'' symbol disappears and a ``master'' label is attached to project path, which means you are successful with your git-commit and git-push.

\begin{figure}
\hspace{-2em}
\begin{minipage}{0.5\textwidth}
\centering
\includegraphics[scale=0.3]{submit-02-add-maven}
\caption{Adding another Maven executable\label{submit-02-add-maven}}
\end{minipage}
\hfill
\begin{minipage}{0.5\textwidth}
\centering
\includegraphics[scale=0.3]{submit-03-add-maven-done}
\caption{Viewing the added external Maven\label{submit-03-add-maven-done}}
\end{minipage}
\hspace{-2em}
\end{figure}

\item Sometimes, the embedded Maven runtime from m2e cannot be succesfully executed to perform a release goal. Therefore, we should add the externally installed Maven runtime into the Eclipse. Click \textbf{Window} (or \textbf{Edit}) $\rightarrow$ \textbf{Preferences} (see Figure \ref{submit-03-add-maven-done}).
\item Select \textbf{Maven} $\rightarrow$ \textbf{Installations}. You will see the ``Embedded'' runtime (as shown in Figure \ref{submit-02-add-maven}). Click \textbf{Add\ldots} to locate the installation path of your Maven runtime.
\item Then, you could see an ``External'' runtime in the installation lists. See Figure \ref{submit-03-add-maven-done}.

\begin{figure}
\centering
\includegraphics[scale=0.3]{submit-04-run-config}
\caption{Getting ready for a release\label{submit-04-run-config}}
\end{figure}

\item Now you can execute a Maven goal within Eclipse (of course, you can also do that outside Eclipse from command line). Click the down-arrow next to the \textbf{Run} button, and select \textbf{Run Configurations\ldots}. See Figure \ref{submit-04-run-config}.

\begin{figure}
\centering
\includegraphics[scale=0.3]{submit-05-run-config-done}
\caption{Configuring Maven goal\label{submit-05-run-config-done}}
\end{figure}

\item In the ``Run Configurations'' window, double-click \textbf{Maven Build} to create a new Maven goal. See Figure \ref{submit-05-run-config-done}. Rename your run configuration name as ``release'' (optional), and click \textbf{Browse Workspace\ldots} to select your project, and type in your goals as follows:

\begin{verbatim}
release:prepare release:perform
\end{verbatim}

This actually defines two goals ``release:prepare'' and ``release:perform''. As you will probably encounter tons of errors during this step, we should review some details of the Maven release.

The Maven guide\footnote{\url{https://maven.apache.org/guides/mini/guide-releasing.html}} tells us what is happening behind these two goals:

\begin{quote}
The release:prepare goal will:

\begin{enumerate}
\item Verify that there are no uncommitted changes in the workspace.
\item Prompt the user for the desired tag, release and development version names.
\item Modify and commit release information into the pom.xml file.
\item Tag the entire project source tree with the new tag name.
\end{enumerate}

The release:perform goal will:

\begin{enumerate}
\item Extract file revisions versioned under the new tag name.
\item Execute the maven build lifecycle on the extracted instance of the project.
\item Deploy the versioned artifacts to appropriate local and remote repositories.
\end{enumerate}
\end{quote}

If the goals are not executed sucessfully, a relatively useful message will be printed out to console to help you discover where the problem is.

\begin{figure}
\centering
\includegraphics[scale=0.3]{submit-06-release}
\caption{A successful release!\label{submit-06-release}}
\end{figure}

\item Finally after you fixed all the problems (if any), you could see the very pleasant ``BUILD SUCCESS'' message in the console (as shown in Figure \ref{submit-06-release}), which means you've done with your Homework 0! Congratulations!

\begin{qa}
\item[Q1] What if I find some bugs after it has been released? How can I resubmit my code?
\item[A1] You can look at the ``Overview'' tab in the Maven POM Editor for your pom.xml file. The version should be ``0.0.2-SNAPSHOT'', which indicates a previous version has been generated. Now, you could redo this task (git-commit, git-push, run Maven goals) to release the version 0.0.2. We will evaluate your code based on the latest release.
\end{qa}

\end{enumerate}


\section{Eclipse (Git, Maven plug-ins integrated)}

If you have an Eclipse IDE for Java Developers with version $\ge$ 3.7, you could probably skip this task. But if you are stuck in a situation where you were told Eclipse is missing a plug-in, you might want to return to this section. If you have other packages (Eclipse Classic or Eclipse for Java EE Developers), please do not skip this task.

\begin{enumerate}
\item Download Eclipse IDE for Java Developers 4.2 at \url{http://www.eclipse.org/downloads/packages/eclipse-ide-java-developers/junor}.

\begin{framed}
\begin{itemize}
\item[Q1] Can I use an older version of Eclipse?
\item[A1] You can try to use an older version, but some plug-ins (e.g., m2e) might complain about the Eclipse version if it is older than 3.7.
\item[Q2] Can I use other Eclipse packages?
\item[A2] Eclipse IDE for Java Developers includes almost all the Eclipse components (jdt, EGit, m2e, and so on) we need for this course. You could also work with other packages, e.g., Eclipse IDE for Jave EE Developers or Eclipse Classics, and as they don't come with such plugins by default, you have to install these plug-ins all by yourself.
\end{itemize}
\end{framed}

\item Install Eclipse by simply uncompressing the downloaded package.

\begin{figure}[t]
\centering
\includegraphics[scale=0.3]{eclipse-01-login}
\caption{Eclipse choose workspace\label{eclipse-01-login}}
\end{figure}

\item Use the default workspace path or create your own workspace, as shown in Figure \ref{eclipse-01-login}. And finally, we could see the Eclipse Welcome view at the end of workspace initialization. See Figure \ref{eclipse-02-welcome}.

\begin{figure}[t]
\centering
\includegraphics[scale=0.3]{eclipse-02-welcome}
\caption{Eclipse Welcome view\label{eclipse-02-welcome}}
\end{figure}

\end{enumerate}


That's the start of your developement. From now on, everthing will become less
platform specific, and we will show you how to configure the workspace, create
your Maven project and release it in the rest of the homework.


\section{A Little Configuration}

\subsection{Letting m2e know your password}

\begin{enumerate}
\item Click \textbf{Edit} (or \textbf{Window} depending on your OS) $\rightarrow$ \textbf{Preferences}, and choose \textbf{Maven} $\rightarrow$ \textbf{User Settings}, and find the default Maven setting path for your system. See Figure \ref{eclipse-03-maven-setting}.

\item Create the \verb|settings.xml| file at the given directory, and copy the text in Listing \ref{settings} into the file, which will store your ID and passwords. Remember to replace \verb|ID| and \verb|PASSWORD| with your personal Maven project repository account we provide you, and also don't upload this file to any remote repository or share it with others.

\item Go back to \textbf{Edit} (or \textbf{Window}) $\rightarrow$ \textbf{Preferences}, and choose \textbf{Maven} $\rightarrow$ \textbf{User Settings} again, you will see the plugin could find the setting file you specified (see Figure \ref{eclipse-04-maven-setting-back}).
% and click \textbf{Update settings}. You will be able to see the repository is now being refreshed. 

\begin{figure}[t]
\hspace{-3em}
\begin{minipage}{0.5\textwidth}
\centering
\includegraphics[scale=0.3]{eclipse-03-maven-setting}
\caption{Eclipse maven user profile setting\label{eclipse-03-maven-setting}}
\end{minipage}
\hfill
\begin{minipage}{0.5\textwidth}
\centering
\includegraphics[scale=0.3]{eclipse-04-maven-setting-back}
\caption{Eclipse maven user profile setting\label{eclipse-04-maven-setting-back}}
\end{minipage}
\hspace{-3em}
\end{figure}

\lstinputlisting[language=XML,float,linewidth=1.1\textwidth,caption=Configuring settings.xml,label=settings]{../lst/settings.xml}

\end{enumerate}

\subsection{Importing Apache UIMA code style template}

To development a software as a team, members should always adopt the same code conventions to improve the readability and maintainability of the project. We suggest you to view the \emph{Code Conventions for the Java Programming Language} at \url{http://www.oracle.com/technetwork/java/codeconv-138413.html}, which was published from Oracle. For our course homeworks, you are required to adopt a set of more specific coding conventions from Apache UIMA project. Details can be found at \url{http://uima.apache.org/codeConventions.html}. At the bottom of the page, you could find a link to download the Eclipse code style template\footnote{\url{http://uima.apache.org/downloads/ApacheUima_EclipseCodeStylePrefs.xml}}.

\begin{enumerate}
\item Download the template and save it in your local filesystem.
\item Click \textbf{Window} $\rightarrow$ \textbf{Preferences}, then go to \textbf{Java} $\rightarrow$ \textbf{Code Style} $\rightarrow$ \textbf{Formatter}, and click \textbf{Import\ldots}. 
\end{enumerate}

Remember before you finish editing a Java file, press \textbf{Ctrl+Shift+F} to perform an automatic code formation.

Another optional but useful tool for you to check your code style is the Eclipse Checkstyle plug-in. You can learn how to download and install the plug-in at \url{http://eclipse-cs.sourceforge.net/}.




\chapter{Installing and Configuring Softwares}

In this task, you are required to install the tools you've already been familiar
with (at least heard about).

\textbf{Important notes:} This might be the hardest part of your Homework 0,
since installation of the same software on different platforms differs a lot,
which means we could not show you detailed steps for each platform. Instead, we
will provide you with links for the installation instructions for you to follow,
and we have created three forums for you to discuss Windows/Linux/Mac platform
related issues. We will try our best to help you solve whatever problems you
might have, but we also encourage you, the experts in any of these platforms, to
work with us to answer the questions from your classmates. You can find the
discussion boards at \url{http://blackboard.andrew.cmu.edu}, and select the
course, then the \textbf{Tools} and \textbf{Discussion Board}.


\section{Installing JDK}

If you have the latest JDK 6 installed\footnote{You should check out the latest
version at
\url{http://www.oracle.com/technetwork/java/javase/downloads/index.html} as the
version number grows really fast}, you could skip this task.

We assume you have experience in Java programming, but we still need to clarify
the Java environment for the course. If you don't have any Java experience,
probably you need to look for a Java textbook. It might also be fine if you
think you have tons of experience in programming in C++/C\# and you feel
confident to learn Java by just reading others' codes and guessing their
meanings. It's up to you!

\begin{enumerate}

\item Visit
\url{http://www.oracle.com/technetwork/java/javase/downloads/index.html}, and
choose the platform you are using to download JDK 6 SE 35\footnote{By the date
of August 31, 2012}.

\begin{qa}

\item[Q1] Can I just install JRE instead of JDK?

\item[A1] No.

\item[Q2] Can I install OpenJDK instead of SunJDK (or OracleJDK)?

\item[A2] Sure, you can. But be aware that sometimes only binary files (aka JRE)
are installed under a folder named \texttt{openjdk-\emph{version}}, rather than
\texttt{openjre-\emph{version}}, which is a bit confusing.

\item[Q3] Can I install JDK 7, 5 or older versions?

\item[A3] You are not recommended to install JDK 7, since you have to modify the
Maven pom file to compile your project, and the cluster that we will run and
test your components does not have JDK 7 set up yet. But it would be fine
(theoretically) if you have just JDK 5 installed, but it is still not
recommended. Versions older than 5 should be completely avoided.

\item[Q4] Can I use an earlier version of JDK 6 (e.g., 6u4)?

\item[A4] It may not put you in a trouble most of time, but in some rare cases,
we did find an exception was thrown due to a bug not in our code but in the
runtime environment. Therefore, we recommend you to upgrade your JDK 6 to the
latest version.

\end{qa}

\item Install JDK from the executable file if available, and set PATH manually
(if you are using a Windows machine). The Java installation page (at
\url{http://www.oracle.com/technetwork/java/javase/index-137561.html}) might be
useful to you.

\end{enumerate}


\section{Commiting and Pushing Your Maven Project Back to Git}

\begin{figure}[t]
\centering
\includegraphics[scale=0.3]{project-19-git-commit}
\caption{Performing a git-commit\label{project-19-git-commit}}
\end{figure}

You should see a ``greater than'' symbol between the project icon and your
project name (see Figure \ref{project-19-git-commit}, which means you have made
some changes to the project so that there exist some differences between your
current workspace version and the branch head. Question marks on the project
element icon represent the corresponding elements are not indexed yet. You might
be wondering what changes you've made because you thought you haven't written
any code. Actually, you have created a Maven project, which process will
generate the \verb|pom.xml| file, and modified the Eclipse configuration files.
So let's perform our first git-commit.

\begin{enumerate}

\item Right-click on the project name, and select \textbf{Team} $\rightarrow$
\textbf{Commit\ldots}. See Figure \ref{project-19-git-commit}, and then type in
your username and email address on GitHub in the popup ``Identify Yourself''
message window.

\begin{figure}[t]
\centering
\includegraphics[scale=0.3]{project-21-git-commit-message}
\caption{Viewing and confirming commit message\label{project-21-git-commit-message}}
\end{figure}

\begin{figure}[t]
\centering
\includegraphics[scale=0.3]{project-22-git-commit-done}
\caption{Committed project\label{project-22-git-commit-done}}
\end{figure}

\item In the ``Commit Changes'' window, you are allowed to choose the files you
want to commit (and also automatically add to the index). As you can see in
Figure \ref{project-21-git-commit-message}, during creating the empty Maven
project, several Eclipse and Maven related configuration files are generated.
Moreover, type in a commit message to describe what changes you've made and
leave your name as well as your email address (which is a convention) in the
committer field. Finally, by clicking \textbf{Commit}, you've done with your
first git-commit. You can see in Figure \ref{project-22-git-commit-done}, the
``greater than'' symbol disappears, and the question marks on the committed
files become a repository icon, which means the files are in the latest version.
You could also find your git-commit helps the code merge into the new master
branch from a NO-HEAD branch.

If you are using SVN, then you've done with sychronizing your local workspace
with the remote repository once you execute a commit command, but remember the
feature of Git? Your project repository is distributed, which means your
previous git-commit conceptually affects all your project repositories, but in
fact you should make it happen with an additional \emph{push}. A nice picture at
\url{http://gitready.com/beginner/2009/01/21/pushing-and-pulling.html} may help
you better understand what is actually happening when you perform git commit,
add, push, fetch, pull, etc.

\begin{figure}[t]
\centering
\includegraphics[scale=0.3]{project-23-git-push}
\caption{Performing a git push\label{project-23-git-push}}
\end{figure}

\item Right-click on the project name, and select \textbf{Team} $\rightarrow$
\textbf{Push to Upstream}. See Figure \ref{project-23-git-push}.

\begin{figure}[t]
\hspace{-2em}
\begin{minipage}{0.5\textwidth}
\centering
\includegraphics[scale=0.3]{project-24-git-push-progress}
\caption{Viewing the git push progress\label{project-24-git-push-progress}}
\end{minipage}
\hfill
\begin{minipage}{0.5\textwidth}
\centering
\includegraphics[scale=0.3]{project-25-git-push-result}
\caption{Viewing the git push result\label{project-25-git-push-result}}
\end{minipage}
\hspace{-1em}
\end{figure}

\item Now you can see a progress indication window pops up (see Figure
\ref{project-24-git-push-progress}), which says \textbf{Pushing to remote
repositories}.

\item Finally, you could see the push results in the ``Push Results'' window.
Click \textbf{OK} to close the window and get back to your workspace.

\end{enumerate}

In your next homework, you will need to use other Git commands within Eclipse.



\chapter{Release to Maven}

In the last task, you will need to submit your Java code by performing a release of your code. But remember that Maven release plug-in will check if all the changes you have made have been checked into the remote repository (i.e. GitHub in our case). So, let's perform a git-commit and a git-push.

\begin{enumerate}

\begin{figure}
\hspace{-1em}
\begin{minipage}{0.5\textwidth}
\centering
\includegraphics[scale=0.3]{simple-code-05-commit}
\caption{Performing a git-commit/push before preparing a release\label{simple-code-05-commit}}
\end{minipage}
\hfill
\begin{minipage}{0.5\textwidth}
\centering
\includegraphics[scale=0.3]{submit-01-preference}
\caption{Starting to add an external Maven executable\label{submit-01-preference}}
\end{minipage}
\hspace{-1em}
\end{figure}

\item Similar to what you did earlier, you execute git-commit and git-push to the project, and you could see the ``greater than'' symbol disappears and a ``master'' label is attached to project path, which means you are successful with your git-commit and git-push.

\begin{figure}
\hspace{-2em}
\begin{minipage}{0.5\textwidth}
\centering
\includegraphics[scale=0.3]{submit-02-add-maven}
\caption{Adding another Maven executable\label{submit-02-add-maven}}
\end{minipage}
\hfill
\begin{minipage}{0.5\textwidth}
\centering
\includegraphics[scale=0.3]{submit-03-add-maven-done}
\caption{Viewing the added external Maven\label{submit-03-add-maven-done}}
\end{minipage}
\hspace{-2em}
\end{figure}

\item Sometimes, the embedded Maven runtime from m2e cannot be succesfully executed to perform a release goal. Therefore, we should add the externally installed Maven runtime into the Eclipse. Click \textbf{Window} (or \textbf{Edit}) $\rightarrow$ \textbf{Preferences} (see Figure \ref{submit-03-add-maven-done}).
\item Select \textbf{Maven} $\rightarrow$ \textbf{Installations}. You will see the ``Embedded'' runtime (as shown in Figure \ref{submit-02-add-maven}). Click \textbf{Add\ldots} to locate the installation path of your Maven runtime.
\item Then, you could see an ``External'' runtime in the installation lists. See Figure \ref{submit-03-add-maven-done}.

\begin{figure}
\centering
\includegraphics[scale=0.3]{submit-04-run-config}
\caption{Getting ready for a release\label{submit-04-run-config}}
\end{figure}

\item Now you can execute a Maven goal within Eclipse (of course, you can also do that outside Eclipse from command line). Click the down-arrow next to the \textbf{Run} button, and select \textbf{Run Configurations\ldots}. See Figure \ref{submit-04-run-config}.

\begin{figure}
\centering
\includegraphics[scale=0.3]{submit-05-run-config-done}
\caption{Configuring Maven goal\label{submit-05-run-config-done}}
\end{figure}

\item In the ``Run Configurations'' window, double-click \textbf{Maven Build} to create a new Maven goal. See Figure \ref{submit-05-run-config-done}. Rename your run configuration name as ``release'' (optional), and click \textbf{Browse Workspace\ldots} to select your project, and type in your goals as follows:

\begin{verbatim}
release:prepare release:perform
\end{verbatim}

This actually defines two goals ``release:prepare'' and ``release:perform''. As you will probably encounter tons of errors during this step, we should review some details of the Maven release.

The Maven guide\footnote{\url{https://maven.apache.org/guides/mini/guide-releasing.html}} tells us what is happening behind these two goals:

\begin{quote}
The release:prepare goal will:

\begin{enumerate}
\item Verify that there are no uncommitted changes in the workspace.
\item Prompt the user for the desired tag, release and development version names.
\item Modify and commit release information into the pom.xml file.
\item Tag the entire project source tree with the new tag name.
\end{enumerate}

The release:perform goal will:

\begin{enumerate}
\item Extract file revisions versioned under the new tag name.
\item Execute the maven build lifecycle on the extracted instance of the project.
\item Deploy the versioned artifacts to appropriate local and remote repositories.
\end{enumerate}
\end{quote}

If the goals are not executed sucessfully, a relatively useful message will be printed out to console to help you discover where the problem is.

\begin{figure}
\centering
\includegraphics[scale=0.3]{submit-06-release}
\caption{A successful release!\label{submit-06-release}}
\end{figure}

\item Finally after you fixed all the problems (if any), you could see the very pleasant ``BUILD SUCCESS'' message in the console (as shown in Figure \ref{submit-06-release}), which means you've done with your Homework 0! Congratulations!

\begin{qa}
\item[Q1] What if I find some bugs after it has been released? How can I resubmit my code?
\item[A1] You can look at the ``Overview'' tab in the Maven POM Editor for your pom.xml file. The version should be ``0.0.2-SNAPSHOT'', which indicates a previous version has been generated. Now, you could redo this task (git-commit, git-push, run Maven goals) to release the version 0.0.2. We will evaluate your code based on the latest release.
\end{qa}

\end{enumerate}


\section{Eclipse (Git, Maven plug-ins integrated)}

If you have an Eclipse IDE for Java Developers with version $\ge$ 3.7, you could probably skip this task. But if you are stuck in a situation where you were told Eclipse is missing a plug-in, you might want to return to this section. If you have other packages (Eclipse Classic or Eclipse for Java EE Developers), please do not skip this task.

\begin{enumerate}
\item Download Eclipse IDE for Java Developers 4.2 at \url{http://www.eclipse.org/downloads/packages/eclipse-ide-java-developers/junor}.

\begin{framed}
\begin{itemize}
\item[Q1] Can I use an older version of Eclipse?
\item[A1] You can try to use an older version, but some plug-ins (e.g., m2e) might complain about the Eclipse version if it is older than 3.7.
\item[Q2] Can I use other Eclipse packages?
\item[A2] Eclipse IDE for Java Developers includes almost all the Eclipse components (jdt, EGit, m2e, and so on) we need for this course. You could also work with other packages, e.g., Eclipse IDE for Jave EE Developers or Eclipse Classics, and as they don't come with such plugins by default, you have to install these plug-ins all by yourself.
\end{itemize}
\end{framed}

\item Install Eclipse by simply uncompressing the downloaded package.

\begin{figure}[t]
\centering
\includegraphics[scale=0.3]{eclipse-01-login}
\caption{Eclipse choose workspace\label{eclipse-01-login}}
\end{figure}

\item Use the default workspace path or create your own workspace, as shown in Figure \ref{eclipse-01-login}. And finally, we could see the Eclipse Welcome view at the end of workspace initialization. See Figure \ref{eclipse-02-welcome}.

\begin{figure}[t]
\centering
\includegraphics[scale=0.3]{eclipse-02-welcome}
\caption{Eclipse Welcome view\label{eclipse-02-welcome}}
\end{figure}

\end{enumerate}


That's the start of your developement. From now on, everthing will become less
platform specific, and we will show you how to configure the workspace, create
your Maven project and release it in the rest of the homework.


\section{A Little Configuration}

\subsection{Letting m2e know your password}

\begin{enumerate}
\item Click \textbf{Edit} (or \textbf{Window} depending on your OS) $\rightarrow$ \textbf{Preferences}, and choose \textbf{Maven} $\rightarrow$ \textbf{User Settings}, and find the default Maven setting path for your system. See Figure \ref{eclipse-03-maven-setting}.

\item Create the \verb|settings.xml| file at the given directory, and copy the text in Listing \ref{settings} into the file, which will store your ID and passwords. Remember to replace \verb|ID| and \verb|PASSWORD| with your personal Maven project repository account we provide you, and also don't upload this file to any remote repository or share it with others.

\item Go back to \textbf{Edit} (or \textbf{Window}) $\rightarrow$ \textbf{Preferences}, and choose \textbf{Maven} $\rightarrow$ \textbf{User Settings} again, you will see the plugin could find the setting file you specified (see Figure \ref{eclipse-04-maven-setting-back}).
% and click \textbf{Update settings}. You will be able to see the repository is now being refreshed. 

\begin{figure}[t]
\hspace{-3em}
\begin{minipage}{0.5\textwidth}
\centering
\includegraphics[scale=0.3]{eclipse-03-maven-setting}
\caption{Eclipse maven user profile setting\label{eclipse-03-maven-setting}}
\end{minipage}
\hfill
\begin{minipage}{0.5\textwidth}
\centering
\includegraphics[scale=0.3]{eclipse-04-maven-setting-back}
\caption{Eclipse maven user profile setting\label{eclipse-04-maven-setting-back}}
\end{minipage}
\hspace{-3em}
\end{figure}

\lstinputlisting[language=XML,float,linewidth=1.1\textwidth,caption=Configuring settings.xml,label=settings]{../lst/settings.xml}

\end{enumerate}

\subsection{Importing Apache UIMA code style template}

To development a software as a team, members should always adopt the same code conventions to improve the readability and maintainability of the project. We suggest you to view the \emph{Code Conventions for the Java Programming Language} at \url{http://www.oracle.com/technetwork/java/codeconv-138413.html}, which was published from Oracle. For our course homeworks, you are required to adopt a set of more specific coding conventions from Apache UIMA project. Details can be found at \url{http://uima.apache.org/codeConventions.html}. At the bottom of the page, you could find a link to download the Eclipse code style template\footnote{\url{http://uima.apache.org/downloads/ApacheUima_EclipseCodeStylePrefs.xml}}.

\begin{enumerate}
\item Download the template and save it in your local filesystem.
\item Click \textbf{Window} $\rightarrow$ \textbf{Preferences}, then go to \textbf{Java} $\rightarrow$ \textbf{Code Style} $\rightarrow$ \textbf{Formatter}, and click \textbf{Import\ldots}. 
\end{enumerate}

Remember before you finish editing a Java file, press \textbf{Ctrl+Shift+F} to perform an automatic code formation.

Another optional but useful tool for you to check your code style is the Eclipse Checkstyle plug-in. You can learn how to download and install the plug-in at \url{http://eclipse-cs.sourceforge.net/}.



\end{document}
