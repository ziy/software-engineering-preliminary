
\section{Importing the Empty Project from GitHub}

The following instruction to guide you create a Maven Git project is similar to a blog post\footnote{\url{http://springinpractice.com/2012/05/06/mavenizing-an-empty-github-project-in-eclipse/}}. In fact, you could create a Maven Git project in various ways. For example, you can create a Maven project first, and share it to Git, or you can just create a Java project, and add Maven nature, then share it to Git, and so on. But, we will show you the way that we think the most comfortable to achieve the goal.

\subsection{Checking out the empty GitHub project}

\begin{enumerate}

\begin{figure}[t]
\hspace{-2em}
\begin{minipage}{0.5\textwidth}
\centering
\includegraphics[scale=0.3]{project-01-import}
\caption{Importing project hosted on GitHub\label{project-01-import}}
\end{minipage}
\hfill
\begin{minipage}{0.5\textwidth}
\centering
\includegraphics[scale=0.3]{project-02-git}
\caption{Choosing Git project\label{project-02-git}}
\end{minipage}
\hspace{-2em}
\end{figure}

\item Click \textbf{File} $\rightarrow$ \textbf{Import\ldots} to get ready to import the empty project hosted on GitHub to local filesystem, as shown in Figure \ref{project-01-import}.
\item Then select \textbf{Git} $\rightarrow$ \textbf{Projects from Git} in the \textbf{Import} window. See Figure \ref{project-02-git}.

\begin{figure}[t]
\hspace{-2em}
\begin{minipage}{0.5\textwidth}
\centering
\includegraphics[scale=0.3]{project-03-git-uri}
\caption{Choosing to import Git project specified by a URI\label{project-03-git-uri}}
\end{minipage}
\hfill
\begin{minipage}{0.5\textwidth}
\centering
\includegraphics[scale=0.3]{project-04-git-repo}
\caption{Typing in your GitHub repository URI\label{project-04-git-repo}}
\end{minipage}
\hspace{-2em}
\end{figure}

\item Select \textbf{URI} as the repository source, then click \textbf{Next}. See Figure \ref{project-03-git-uri}.
\item Copy the URI from your GitHub repository page (c.f. Figure \ref{git-03-uri}), and paste it here as the URI, and the Host and Repository path will be automatically generated for you. You also need to type in your Git Repository username and password, and probably you want to \textbf{Store in in Secure Store}. So just check the box at the bottom of the window, and then click \textbf{Next}. See Figure \ref{project-04-git-repo}.

\begin{figure}[t]
\hspace{-2em}
\begin{minipage}{0.5\textwidth}
\centering
\includegraphics[scale=0.3]{project-05a-git-empty}
\caption{Choosing master branch\label{project-05a-git-empty}}
\end{minipage}
\hfill
\begin{minipage}{0.5\textwidth}
\centering
\includegraphics[scale=0.3]{project-06a-git-local}
\caption{Choosing local Git project location\label{project-06a-git-local}}
\end{minipage}
\hspace{-2em}
\end{figure}

\item Most of the time, you will see EGit plug-in is able to detect the branch \textbf{master} does exist in the repository (as shown in Figure \ref{project-05a-git-empty}). Click \textbf{Next} to proceed to the next step where you can specificy the local storage location for the project. If you are not sure what it means by a local storage, you should go back to Task 1 to learn one of the most important features of Git - distributed version control. Type in your local storage path in \textbf{Directory}, and \textbf{master} will be selected by default as the \textbf{Initial branch} as shown in Figure \ref{project-06a-git-local}. Now you could proceed to next step. 

\begin{figure}[t]
\hspace{-2em}
\begin{minipage}{0.5\textwidth}
\centering
\includegraphics[scale=0.3]{project-05-git-empty}
\caption{Failed to choose master branch\label{project-05-git-empty}}
\end{minipage}
\hfill
\begin{minipage}{0.5\textwidth}
\centering
\includegraphics[scale=0.3]{project-06-git-local}
\caption{Choosing local Git storage location for the project\label{project-06-git-local}}
\end{minipage}
\hspace{-2em}
\end{figure}

\begin{qa}
\item[Q1] If the master branch cannot be found automatically by EGit, i.e., Figure \ref{project-05-git-empty} is shown, where it says \textbf{Source Git repository is empty} and no branch can be selected, instead of Figure \ref{project-05a-git-empty}. What should I do?
\item[A1] No problem. You could still click \textbf{Next} to proceed, and in the next window, you can still assign a local storage for the project, but you cannot specify the \textbf{Initial branch} (see Figure \ref{project-06-git-local}). You can specify the master branch at a later stage when we try to perform your first git-commit.
\end{qa}

\item Finally, if you chose to store the password in a secure store, you need to type in (of course twice) a master password for the secure storage. Note that this is not your Maven or GitHub password, and it is only used within the scope of Eclipse. See Figure \ref{project-07-git-password}.

\begin{figure}[t]
\centering
\includegraphics[scale=0.3]{project-07-git-password}
\caption{Typing in your password\label{project-07-git-password}}
\end{figure}

\end{enumerate}

\subsection{Creating a Maven project}

So far, the empty project has been checked into the local filesystem. At this point, files that Eclipse should rely on, such as \verb|.project|, have not been created, in other words, we haven't set up an Eclipse project yet.

\begin{enumerate}

\begin{figure}[t]
\hspace{-2em}
\begin{minipage}{0.5\textwidth}
\includegraphics[scale=0.3]{project-08-new-project}
\caption{Creating new project\label{project-08-new-project}}
\end{minipage}
\hfill
\begin{minipage}{0.5\textwidth}
\centering
\includegraphics[scale=0.3]{project-09-maven}
\caption{Choosing to create new Maven project\label{project-09-maven}}
\end{minipage}
\hspace{-2em}
\end{figure}

\item Now, we select \textbf{Use the New Project wizard} in the ``Project Import'' window and click \textbf{Finish} to begin create an Eclipse Maven Project, as you can see in Figure \ref{project-08-new-project}.
\item Then select \textbf{Maven} $\rightarrow$ \textbf{Maven Project}, then click \textbf{Next} to continue. See Figure \ref{project-09-maven}.

\begin{figure}[t]
\hspace{-3em}
\begin{minipage}{0.5\textwidth}
\centering
\includegraphics[scale=0.3]{project-10-maven-location}
\caption{Specifying Maven Project location\label{project-10-maven-location}}
\end{minipage}
\hfill
\begin{minipage}{0.5\textwidth}
\centering
\includegraphics[scale=0.3]{project-11-maven-artifact}
\caption{Typing in basic information for Maven artifact\label{project-11-maven-artifact}}
\end{minipage}
\hspace{-3em}
\end{figure}

\item In the ``New Maven Project'' window, you select both \textbf{Create a simple project (skip archetype selection)} and \textbf{Use default Workspace location}. See Figure \ref{project-10-maven-location}. If you have already forgotten what archetype and artifact are, you really need to go back to Task 1 to review the basic concepts of Maven. For Homework 0, you will not need an archetype to base on, but you will soon realize that Maven archetypes are crucial to software development. We will provide you the archetype you need for Homework 1 and 2 (of course, via our course Maven repository).
\item Then, in the next window, you need to type in the \textbf{Group Id} and \textbf{Artifact Id} for the artifact you are going to create. The Group Id for Homework 0 is

\begin{center}
\textbf{edu.cmu.lti.11791.f12.hw0}
\end{center}

and the Artifact Id should be

\begin{center}
\textbf{hw0-ANDREW\_ID}
\end{center}

Replace \textbf{ANDREW\_ID} with your Andrew ID. You must type in the correct information to create your artifact, since both Group Id and Artifact Id will be used to generate the jar file and eventually submit the jar to the correct folder in our Maven repository. You do not need to change the rest of the configuration, and press \textbf{Finish} to complete creating the Maven project. On some platforms\footnote{For example, Windows XP/Helios as reported by Martin van Velsen}, you will need to fill in a Name and Description otherwise the wizard will generate an error.

\begin{figure}[t]
\centering
\includegraphics[scale=0.3]{project-12-jre}
\caption{Configuring Build Path\label{project-12-jre}}
\end{figure}

\item If you find the project is built based on an earlier version of Java (e.g., JDK 1.5), then it is time to change it to JDK 6. If JDK environment that your project is using is 1.6, then you can skip this additional step. First, right-click on the project name (i.e. hw0-ziy) in the ``Package Explorer'' view, which is usually located on the left of your workspace. Then select \textbf{Build Path} $\rightarrow$ \textbf{Configure Build Path\ldots}, as shown in Figure \ref{project-12-jre}.

\begin{figure}[t]
\centering
\includegraphics[scale=0.3]{project-13-jre-view}
\caption{Viewing JRE version\label{project-13-jre-view}}
\end{figure}

\item Then click the \textbf{Libraries} tab, select \textbf{JRE System Library}, and click \textbf{Edit}. See Figure \ref{project-13-jre-view}.

\begin{figure}[t]
\centering
\includegraphics[scale=0.3]{project-14-jre-change}
\caption{Choosing a different JRE version\label{project-14-jre-change}}
\end{figure}

\item In the popup window, select any Jave 1.6 environment you have installed. Note that the options in your workspace might be totally different from what you see in Figure \ref{project-14-jre-change}.

\end{enumerate}

\subsection{Sharing your project via Git}

If you encountered the problem we described in Figure \ref{project-05-git-empty} and \ref{project-06-git-local}, you need to re-establish the connection from the local Git repository to the repository you built on GitHub. If you do not have such issue, you can skip this part.

\begin{figure}[t]
\centering
\includegraphics[scale=0.3]{project-15-share}
\caption{Sharing project again\label{project-15-share}}
\end{figure}

You need to right-click on the project name, and select \textbf{Team} $\rightarrow$ \textbf{Share Project\ldots} in order to share your project on GitHub. See Figure \ref{project-15-share}. You should be able to proceed by selecting the repository type (Git). If you still find come across the NO-HEAD issue (see Figure \ref{project-18-git-nohead}) after you share your project, don't panic as you can specify the branch during your first git-commit.

\begin{figure}[t]
\centering
\includegraphics[scale=0.3]{project-18-git-nohead}
\caption{Nohead issue\label{project-18-git-nohead}}
\end{figure}

