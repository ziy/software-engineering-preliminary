% 
% This work is licensed under the Creative Commons Attribution-ShareAlike 3.0
% Unported License. To view a copy of this license, visit
% http://creativecommons.org/licenses/by-sa/3.0/.
%
% author: Zi Yang <ziy@cs.cmu.edu>
% date: 09/02/2012
%
\documentclass[oneside]{memoir}



\renewcommand{\chaptername}{Task}

\usepackage{times}

\usepackage{titlesec}
\titleformat{\section}{\normalfont\Large\bfseries}{Task \thesection}{1em}{}

\usepackage{url}
\usepackage{hyperref}

\usepackage{graphicx}
\graphicspath{{../../fig/uml-and-uima/}}

\usepackage{pstricks}
\usepackage{epsfig}

\usepackage{listings}
\usepackage{color}
\definecolor{dkgreen}{rgb}{0,0.6,0}
\definecolor{gray}{rgb}{0.5,0.5,0.5}
\definecolor{lightblue}{rgb}{0.95,0.95,1}
\definecolor{mauve}{rgb}{0.58,0,0.82}
\lstset{
  basicstyle=\ttfamily\small,
  numbers=left,
  numberstyle=\tiny\color{gray},
  stepnumber=2, 
  numbersep=5pt,
  backgroundcolor=\color{lightblue},
  showspaces=false,
  showstringspaces=false,
  showtabs=false,
  frame=lines,
  rulecolor=\color{black},
  tabsize=2,
  captionpos=b,
  breaklines=true,
  breakatwhitespace=false,
  title=\lstname,
  keywordstyle=\color{blue},
  commentstyle=\color{dkgreen},
  stringstyle=\color{mauve},
  escapeinside={\%*}{*)},
  morekeywords={*,...},
}
\usepackage{letltxmacro}
\makeatletter
\LetLtxMacro\@@lst@inputlisting\lst@inputlisting
\renewcommand\lst@inputlisting[2][]{%
  \try@listingspath{#2}%
  \if@tempswa
    \typeout{Using \@foundlisting}%
    \@@lst@inputlisting[#1]{\@foundlisting}%
  \else
    \typeout{Missing file #2}\endgroup
  \fi}
\def\listingspath#1{\def\@listingspath{#1}}
\listingspath{}
\def\try@listingspath#1{%
  \@tempswafalse
  \expandafter\@tfor\expandafter\next
  \expandafter:\expandafter=\@listingspath\do
  {\if@tempswa\@break@tfor\fi
   \IfFileExists{\next/#1}{\@tempswatrue\xdef\@foundlisting{\next/#1}}{}}%
}
\makeatletter
\listingspath{{../../lst/uml-and-uima/}}

\usepackage{multirow}

\definecolor{shadecolor}{gray}{0.9}

\newenvironment{qa}
{\begin{shaded}\begin{itemize}}
{\end{itemize}\end{shaded}}

\title{{\bfseries 11-791 Design and Engineering of Intelligent Information
System \& \\11-693 Software Methods for Biotechnology \\Homework 1}\\
\vspace{1em}
\itshape\rmfamily Logical Data Model and UIMA Type System Design \&
Implementation}

\date{}

\begin{document}

\begin{titlingpage}
\maketitle

\hspace{-0.1\textwidth}
\begin{minipage}{1.2\textwidth}
\vspace{-5em}
\textbf{Important dates}
\begin{itemize}

\item \textbf{Hand out: September 4.}\footnote{This version was built on \today}

\item \textbf{Turn in: September 11.} In this homework, you will create a
logical data model for a sample information processing task. Based on your
analysis of the requirements, you will need to submit a report to describe how
you design an appropriate UIMA type system to model the required information
types. Then, you will use the UIMA tooling in Eclipse to create a type system
\verb|.xml| file; use the \texttt{JCasGen} plugin to compile your type system
into Java type classes.

If you have ever looked into your \verb|target/| directory or Maven
course-release repository for Homework 0, then you will find that Maven will
package the binary files, source files, and Javadocs into different jars, and
deploy them on the server, which means all you need to do is to put all the
source codes and descriptors, as well as the documents, in the right place of
your hw1-ID project. You should organize your project in the same hierarchy as
shown below\footnote{To simplify your submission process and our evaluation
process, we ask you to create \texttt{src/main/resources/docs} for your
documents. But you should keep in mind it is NOT a good practise to have
documentation within a \texttt{src} folder.}:

\small
\begin{verbatim}
hw1-ID
|- pom.xml
`- src
     `- main
        |- java
      |   `- **/*.java      /* Java classes generated by JCasGen */ 
        `- resources
         |- hw1-ID-typesystem.xml /* your type system */
         |- **/*.*                /* (optional) other resources */
         `- docs
            `- hw1-ID-report.pdf  /* your report for design */

\end{verbatim}
\normalsize

\end{itemize}

\end{minipage}
\hspace{-0.1\textwidth}

\hspace{-0.1\textwidth}
\begin{minipage}{1.2\textwidth}

Several notes about organizing your Maven project and other additional
information:

\begin{enumerate}

\item \textbf{Submission:} The same way as you did for Homework 0 (set up GitHub
repo, create Maven project, write your code, submit to Maven repo), except that
the name has changed to hw1-ID.

\item \textbf{Your source files and type system descriptor(s):} \verb|**/*.java|
and \verb|**/*.*| tell you that you don't need to flatten your folder hierarchy,
instead we encourage you to place your Java codes in the right package by
specifying the right package name for each type you defined in the
\verb|hw1-ID-typesystem.xml| file. You can also create sub type systems and
import them into your main type system. We will look into your jar packages.

\item \textbf{Your report for design:} We expect to see your requirement
analysis for the sample information system in your report. We will pull out your
documents from your jar files. Remember to include your ID as part of file
names, and put your name and ID in your document. Please submit in PDF format
only. If you think it might be easier to describe your design with a UML
diagram, you can optionally put one in your report.

\item \textbf{Javadocs:} Please remember to give an appropriate description for
each type you create and feature you add, which will end up being the Java
comments in the Java classes generated by the \texttt{JCasGen}. You might want
to refer to any best practise (yes, there might be more than one) to write your
Javadoc, e.g.,
\url{http://www.oracle.com/technetwork/java/javase/documentation/index-137868.html}.
We expect you to describe your major components at the class level of your
Javadoc, and put additional comments on the most important methods if any.

\end{enumerate}

\textbf{Useful information}
\begin{enumerate}
\item We encourage you to start Homework 1 as early as possible, especially if
you haven't got a chance to browse the UIMA portal.

\item Please post your questions regarding Homework 1 on Piazza
\url{https://piazza.com/cmu/fall2013/11791}. For other issues or concenrs, you
can also send us mails to, Alkesh Patel
(\href{mailto:alkeshku@andrew.cmu.edu}{\nolinkurl{alkeshku@andrew.cmu.edu}}), Di
Wang (\href{mailto:diwang@cs.cmu.edu}{\nolinkurl{diwang@cs.cmu.edu}}), or Zi
Yang (\href{mailto:ziy@cs.cmu.edu}{\nolinkurl{ziy@cs.cmu.edu}}).

\item Again, both source files and pdf file of this assignment are
publicly available on my GitHub

\url{http://github.com/ziy/software-engineering-preliminary}

Please feel free to fork the project and send a pull request back to me as some
of you did for Homework 0 for any error. Or you can just report an issue at

\url{http://github.com/ziy/software-engineering-preliminary/issues}

\end{enumerate}

\end{minipage}
\hspace{-0.1\textwidth}

\end{titlingpage}


\chapter{Installing and Configuring Softwares}

In this task, you are required to install the tools you've already been familiar
with (at least heard about).

\textbf{Important notes:} This might be the hardest part of your Homework 0,
since installation of the same software on different platforms differs a lot,
which means we could not show you detailed steps for each platform. Instead, we
will provide you with links for the installation instructions for you to follow,
and we have created three forums for you to discuss Windows/Linux/Mac platform
related issues. We will try our best to help you solve whatever problems you
might have, but we also encourage you, the experts in any of these platforms, to
work with us to answer the questions from your classmates. You can find the
discussion boards at \url{http://blackboard.andrew.cmu.edu}, and select the
course, then the \textbf{Tools} and \textbf{Discussion Board}.


\section{Installing JDK}

If you have the latest JDK 6 installed\footnote{You should check out the latest
version at
\url{http://www.oracle.com/technetwork/java/javase/downloads/index.html} as the
version number grows really fast}, you could skip this task.

We assume you have experience in Java programming, but we still need to clarify
the Java environment for the course. If you don't have any Java experience,
probably you need to look for a Java textbook. It might also be fine if you
think you have tons of experience in programming in C++/C\# and you feel
confident to learn Java by just reading others' codes and guessing their
meanings. It's up to you!

\begin{enumerate}

\item Visit
\url{http://www.oracle.com/technetwork/java/javase/downloads/index.html}, and
choose the platform you are using to download JDK 6 SE 35\footnote{By the date
of August 31, 2012}.

\begin{qa}

\item[Q1] Can I just install JRE instead of JDK?

\item[A1] No.

\item[Q2] Can I install OpenJDK instead of SunJDK (or OracleJDK)?

\item[A2] Sure, you can. But be aware that sometimes only binary files (aka JRE)
are installed under a folder named \texttt{openjdk-\emph{version}}, rather than
\texttt{openjre-\emph{version}}, which is a bit confusing.

\item[Q3] Can I install JDK 7, 5 or older versions?

\item[A3] You are not recommended to install JDK 7, since you have to modify the
Maven pom file to compile your project, and the cluster that we will run and
test your components does not have JDK 7 set up yet. But it would be fine
(theoretically) if you have just JDK 5 installed, but it is still not
recommended. Versions older than 5 should be completely avoided.

\item[Q4] Can I use an earlier version of JDK 6 (e.g., 6u4)?

\item[A4] It may not put you in a trouble most of time, but in some rare cases,
we did find an exception was thrown due to a bug not in our code but in the
runtime environment. Therefore, we recommend you to upgrade your JDK 6 to the
latest version.

\end{qa}

\item Install JDK from the executable file if available, and set PATH manually
(if you are using a Windows machine). The Java installation page (at
\url{http://www.oracle.com/technetwork/java/javase/index-137561.html}) might be
useful to you.

\end{enumerate}


\section{Commiting and Pushing Your Maven Project Back to Git}

\begin{figure}[t]
\centering
\includegraphics[scale=0.3]{project-19-git-commit}
\caption{Performing a git-commit\label{project-19-git-commit}}
\end{figure}

You should see a ``greater than'' symbol between the project icon and your
project name (see Figure \ref{project-19-git-commit}, which means you have made
some changes to the project so that there exist some differences between your
current workspace version and the branch head. Question marks on the project
element icon represent the corresponding elements are not indexed yet. You might
be wondering what changes you've made because you thought you haven't written
any code. Actually, you have created a Maven project, which process will
generate the \verb|pom.xml| file, and modified the Eclipse configuration files.
So let's perform our first git-commit.

\begin{enumerate}

\item Right-click on the project name, and select \textbf{Team} $\rightarrow$
\textbf{Commit\ldots}. See Figure \ref{project-19-git-commit}, and then type in
your username and email address on GitHub in the popup ``Identify Yourself''
message window.

\begin{figure}[t]
\centering
\includegraphics[scale=0.3]{project-21-git-commit-message}
\caption{Viewing and confirming commit message\label{project-21-git-commit-message}}
\end{figure}

\begin{figure}[t]
\centering
\includegraphics[scale=0.3]{project-22-git-commit-done}
\caption{Committed project\label{project-22-git-commit-done}}
\end{figure}

\item In the ``Commit Changes'' window, you are allowed to choose the files you
want to commit (and also automatically add to the index). As you can see in
Figure \ref{project-21-git-commit-message}, during creating the empty Maven
project, several Eclipse and Maven related configuration files are generated.
Moreover, type in a commit message to describe what changes you've made and
leave your name as well as your email address (which is a convention) in the
committer field. Finally, by clicking \textbf{Commit}, you've done with your
first git-commit. You can see in Figure \ref{project-22-git-commit-done}, the
``greater than'' symbol disappears, and the question marks on the committed
files become a repository icon, which means the files are in the latest version.
You could also find your git-commit helps the code merge into the new master
branch from a NO-HEAD branch.

If you are using SVN, then you've done with sychronizing your local workspace
with the remote repository once you execute a commit command, but remember the
feature of Git? Your project repository is distributed, which means your
previous git-commit conceptually affects all your project repositories, but in
fact you should make it happen with an additional \emph{push}. A nice picture at
\url{http://gitready.com/beginner/2009/01/21/pushing-and-pulling.html} may help
you better understand what is actually happening when you perform git commit,
add, push, fetch, pull, etc.

\begin{figure}[t]
\centering
\includegraphics[scale=0.3]{project-23-git-push}
\caption{Performing a git push\label{project-23-git-push}}
\end{figure}

\item Right-click on the project name, and select \textbf{Team} $\rightarrow$
\textbf{Push to Upstream}. See Figure \ref{project-23-git-push}.

\begin{figure}[t]
\hspace{-2em}
\begin{minipage}{0.5\textwidth}
\centering
\includegraphics[scale=0.3]{project-24-git-push-progress}
\caption{Viewing the git push progress\label{project-24-git-push-progress}}
\end{minipage}
\hfill
\begin{minipage}{0.5\textwidth}
\centering
\includegraphics[scale=0.3]{project-25-git-push-result}
\caption{Viewing the git push result\label{project-25-git-push-result}}
\end{minipage}
\hspace{-1em}
\end{figure}

\item Now you can see a progress indication window pops up (see Figure
\ref{project-24-git-push-progress}), which says \textbf{Pushing to remote
repositories}.

\item Finally, you could see the push results in the ``Push Results'' window.
Click \textbf{OK} to close the window and get back to your workspace.

\end{enumerate}

In your next homework, you will need to use other Git commands within Eclipse.



\chapter{Release to Maven}

In the last task, you will need to submit your Java code by performing a release of your code. But remember that Maven release plug-in will check if all the changes you have made have been checked into the remote repository (i.e. GitHub in our case). So, let's perform a git-commit and a git-push.

\begin{enumerate}

\begin{figure}
\hspace{-1em}
\begin{minipage}{0.5\textwidth}
\centering
\includegraphics[scale=0.3]{simple-code-05-commit}
\caption{Performing a git-commit/push before preparing a release\label{simple-code-05-commit}}
\end{minipage}
\hfill
\begin{minipage}{0.5\textwidth}
\centering
\includegraphics[scale=0.3]{submit-01-preference}
\caption{Starting to add an external Maven executable\label{submit-01-preference}}
\end{minipage}
\hspace{-1em}
\end{figure}

\item Similar to what you did earlier, you execute git-commit and git-push to the project, and you could see the ``greater than'' symbol disappears and a ``master'' label is attached to project path, which means you are successful with your git-commit and git-push.

\begin{figure}
\hspace{-2em}
\begin{minipage}{0.5\textwidth}
\centering
\includegraphics[scale=0.3]{submit-02-add-maven}
\caption{Adding another Maven executable\label{submit-02-add-maven}}
\end{minipage}
\hfill
\begin{minipage}{0.5\textwidth}
\centering
\includegraphics[scale=0.3]{submit-03-add-maven-done}
\caption{Viewing the added external Maven\label{submit-03-add-maven-done}}
\end{minipage}
\hspace{-2em}
\end{figure}

\item Sometimes, the embedded Maven runtime from m2e cannot be succesfully executed to perform a release goal. Therefore, we should add the externally installed Maven runtime into the Eclipse. Click \textbf{Window} (or \textbf{Edit}) $\rightarrow$ \textbf{Preferences} (see Figure \ref{submit-03-add-maven-done}).
\item Select \textbf{Maven} $\rightarrow$ \textbf{Installations}. You will see the ``Embedded'' runtime (as shown in Figure \ref{submit-02-add-maven}). Click \textbf{Add\ldots} to locate the installation path of your Maven runtime.
\item Then, you could see an ``External'' runtime in the installation lists. See Figure \ref{submit-03-add-maven-done}.

\begin{figure}
\centering
\includegraphics[scale=0.3]{submit-04-run-config}
\caption{Getting ready for a release\label{submit-04-run-config}}
\end{figure}

\item Now you can execute a Maven goal within Eclipse (of course, you can also do that outside Eclipse from command line). Click the down-arrow next to the \textbf{Run} button, and select \textbf{Run Configurations\ldots}. See Figure \ref{submit-04-run-config}.

\begin{figure}
\centering
\includegraphics[scale=0.3]{submit-05-run-config-done}
\caption{Configuring Maven goal\label{submit-05-run-config-done}}
\end{figure}

\item In the ``Run Configurations'' window, double-click \textbf{Maven Build} to create a new Maven goal. See Figure \ref{submit-05-run-config-done}. Rename your run configuration name as ``release'' (optional), and click \textbf{Browse Workspace\ldots} to select your project, and type in your goals as follows:

\begin{verbatim}
release:prepare release:perform
\end{verbatim}

This actually defines two goals ``release:prepare'' and ``release:perform''. As you will probably encounter tons of errors during this step, we should review some details of the Maven release.

The Maven guide\footnote{\url{https://maven.apache.org/guides/mini/guide-releasing.html}} tells us what is happening behind these two goals:

\begin{quote}
The release:prepare goal will:

\begin{enumerate}
\item Verify that there are no uncommitted changes in the workspace.
\item Prompt the user for the desired tag, release and development version names.
\item Modify and commit release information into the pom.xml file.
\item Tag the entire project source tree with the new tag name.
\end{enumerate}

The release:perform goal will:

\begin{enumerate}
\item Extract file revisions versioned under the new tag name.
\item Execute the maven build lifecycle on the extracted instance of the project.
\item Deploy the versioned artifacts to appropriate local and remote repositories.
\end{enumerate}
\end{quote}

If the goals are not executed sucessfully, a relatively useful message will be printed out to console to help you discover where the problem is.

\begin{figure}
\centering
\includegraphics[scale=0.3]{submit-06-release}
\caption{A successful release!\label{submit-06-release}}
\end{figure}

\item Finally after you fixed all the problems (if any), you could see the very pleasant ``BUILD SUCCESS'' message in the console (as shown in Figure \ref{submit-06-release}), which means you've done with your Homework 0! Congratulations!

\begin{qa}
\item[Q1] What if I find some bugs after it has been released? How can I resubmit my code?
\item[A1] You can look at the ``Overview'' tab in the Maven POM Editor for your pom.xml file. The version should be ``0.0.2-SNAPSHOT'', which indicates a previous version has been generated. Now, you could redo this task (git-commit, git-push, run Maven goals) to release the version 0.0.2. We will evaluate your code based on the latest release.
\end{qa}

\end{enumerate}


\section{Eclipse (Git, Maven plug-ins integrated)}

If you have an Eclipse IDE for Java Developers with version $\ge$ 3.7, you could probably skip this task. But if you are stuck in a situation where you were told Eclipse is missing a plug-in, you might want to return to this section. If you have other packages (Eclipse Classic or Eclipse for Java EE Developers), please do not skip this task.

\begin{enumerate}
\item Download Eclipse IDE for Java Developers 4.2 at \url{http://www.eclipse.org/downloads/packages/eclipse-ide-java-developers/junor}.

\begin{framed}
\begin{itemize}
\item[Q1] Can I use an older version of Eclipse?
\item[A1] You can try to use an older version, but some plug-ins (e.g., m2e) might complain about the Eclipse version if it is older than 3.7.
\item[Q2] Can I use other Eclipse packages?
\item[A2] Eclipse IDE for Java Developers includes almost all the Eclipse components (jdt, EGit, m2e, and so on) we need for this course. You could also work with other packages, e.g., Eclipse IDE for Jave EE Developers or Eclipse Classics, and as they don't come with such plugins by default, you have to install these plug-ins all by yourself.
\end{itemize}
\end{framed}

\item Install Eclipse by simply uncompressing the downloaded package.

\begin{figure}[t]
\centering
\includegraphics[scale=0.3]{eclipse-01-login}
\caption{Eclipse choose workspace\label{eclipse-01-login}}
\end{figure}

\item Use the default workspace path or create your own workspace, as shown in Figure \ref{eclipse-01-login}. And finally, we could see the Eclipse Welcome view at the end of workspace initialization. See Figure \ref{eclipse-02-welcome}.

\begin{figure}[t]
\centering
\includegraphics[scale=0.3]{eclipse-02-welcome}
\caption{Eclipse Welcome view\label{eclipse-02-welcome}}
\end{figure}

\end{enumerate}


That's the start of your developement. From now on, everthing will become less
platform specific, and we will show you how to configure the workspace, create
your Maven project and release it in the rest of the homework.


\section{A Little Configuration}

\subsection{Letting m2e know your password}

\begin{enumerate}
\item Click \textbf{Edit} (or \textbf{Window} depending on your OS) $\rightarrow$ \textbf{Preferences}, and choose \textbf{Maven} $\rightarrow$ \textbf{User Settings}, and find the default Maven setting path for your system. See Figure \ref{eclipse-03-maven-setting}.

\item Create the \verb|settings.xml| file at the given directory, and copy the text in Listing \ref{settings} into the file, which will store your ID and passwords. Remember to replace \verb|ID| and \verb|PASSWORD| with your personal Maven project repository account we provide you, and also don't upload this file to any remote repository or share it with others.

\item Go back to \textbf{Edit} (or \textbf{Window}) $\rightarrow$ \textbf{Preferences}, and choose \textbf{Maven} $\rightarrow$ \textbf{User Settings} again, you will see the plugin could find the setting file you specified (see Figure \ref{eclipse-04-maven-setting-back}).
% and click \textbf{Update settings}. You will be able to see the repository is now being refreshed. 

\begin{figure}[t]
\hspace{-3em}
\begin{minipage}{0.5\textwidth}
\centering
\includegraphics[scale=0.3]{eclipse-03-maven-setting}
\caption{Eclipse maven user profile setting\label{eclipse-03-maven-setting}}
\end{minipage}
\hfill
\begin{minipage}{0.5\textwidth}
\centering
\includegraphics[scale=0.3]{eclipse-04-maven-setting-back}
\caption{Eclipse maven user profile setting\label{eclipse-04-maven-setting-back}}
\end{minipage}
\hspace{-3em}
\end{figure}

\lstinputlisting[language=XML,float,linewidth=1.1\textwidth,caption=Configuring settings.xml,label=settings]{../lst/settings.xml}

\end{enumerate}

\subsection{Importing Apache UIMA code style template}

To development a software as a team, members should always adopt the same code conventions to improve the readability and maintainability of the project. We suggest you to view the \emph{Code Conventions for the Java Programming Language} at \url{http://www.oracle.com/technetwork/java/codeconv-138413.html}, which was published from Oracle. For our course homeworks, you are required to adopt a set of more specific coding conventions from Apache UIMA project. Details can be found at \url{http://uima.apache.org/codeConventions.html}. At the bottom of the page, you could find a link to download the Eclipse code style template\footnote{\url{http://uima.apache.org/downloads/ApacheUima_EclipseCodeStylePrefs.xml}}.

\begin{enumerate}
\item Download the template and save it in your local filesystem.
\item Click \textbf{Window} $\rightarrow$ \textbf{Preferences}, then go to \textbf{Java} $\rightarrow$ \textbf{Code Style} $\rightarrow$ \textbf{Formatter}, and click \textbf{Import\ldots}. 
\end{enumerate}

Remember before you finish editing a Java file, press \textbf{Ctrl+Shift+F} to perform an automatic code formation.

Another optional but useful tool for you to check your code style is the Eclipse Checkstyle plug-in. You can learn how to download and install the plug-in at \url{http://eclipse-cs.sourceforge.net/}.




\chapter{Installing and Configuring Softwares}

In this task, you are required to install the tools you've already been familiar
with (at least heard about).

\textbf{Important notes:} This might be the hardest part of your Homework 0,
since installation of the same software on different platforms differs a lot,
which means we could not show you detailed steps for each platform. Instead, we
will provide you with links for the installation instructions for you to follow,
and we have created three forums for you to discuss Windows/Linux/Mac platform
related issues. We will try our best to help you solve whatever problems you
might have, but we also encourage you, the experts in any of these platforms, to
work with us to answer the questions from your classmates. You can find the
discussion boards at \url{http://blackboard.andrew.cmu.edu}, and select the
course, then the \textbf{Tools} and \textbf{Discussion Board}.


\section{Installing JDK}

If you have the latest JDK 6 installed\footnote{You should check out the latest
version at
\url{http://www.oracle.com/technetwork/java/javase/downloads/index.html} as the
version number grows really fast}, you could skip this task.

We assume you have experience in Java programming, but we still need to clarify
the Java environment for the course. If you don't have any Java experience,
probably you need to look for a Java textbook. It might also be fine if you
think you have tons of experience in programming in C++/C\# and you feel
confident to learn Java by just reading others' codes and guessing their
meanings. It's up to you!

\begin{enumerate}

\item Visit
\url{http://www.oracle.com/technetwork/java/javase/downloads/index.html}, and
choose the platform you are using to download JDK 6 SE 35\footnote{By the date
of August 31, 2012}.

\begin{qa}

\item[Q1] Can I just install JRE instead of JDK?

\item[A1] No.

\item[Q2] Can I install OpenJDK instead of SunJDK (or OracleJDK)?

\item[A2] Sure, you can. But be aware that sometimes only binary files (aka JRE)
are installed under a folder named \texttt{openjdk-\emph{version}}, rather than
\texttt{openjre-\emph{version}}, which is a bit confusing.

\item[Q3] Can I install JDK 7, 5 or older versions?

\item[A3] You are not recommended to install JDK 7, since you have to modify the
Maven pom file to compile your project, and the cluster that we will run and
test your components does not have JDK 7 set up yet. But it would be fine
(theoretically) if you have just JDK 5 installed, but it is still not
recommended. Versions older than 5 should be completely avoided.

\item[Q4] Can I use an earlier version of JDK 6 (e.g., 6u4)?

\item[A4] It may not put you in a trouble most of time, but in some rare cases,
we did find an exception was thrown due to a bug not in our code but in the
runtime environment. Therefore, we recommend you to upgrade your JDK 6 to the
latest version.

\end{qa}

\item Install JDK from the executable file if available, and set PATH manually
(if you are using a Windows machine). The Java installation page (at
\url{http://www.oracle.com/technetwork/java/javase/index-137561.html}) might be
useful to you.

\end{enumerate}


\section{Commiting and Pushing Your Maven Project Back to Git}

\begin{figure}[t]
\centering
\includegraphics[scale=0.3]{project-19-git-commit}
\caption{Performing a git-commit\label{project-19-git-commit}}
\end{figure}

You should see a ``greater than'' symbol between the project icon and your
project name (see Figure \ref{project-19-git-commit}, which means you have made
some changes to the project so that there exist some differences between your
current workspace version and the branch head. Question marks on the project
element icon represent the corresponding elements are not indexed yet. You might
be wondering what changes you've made because you thought you haven't written
any code. Actually, you have created a Maven project, which process will
generate the \verb|pom.xml| file, and modified the Eclipse configuration files.
So let's perform our first git-commit.

\begin{enumerate}

\item Right-click on the project name, and select \textbf{Team} $\rightarrow$
\textbf{Commit\ldots}. See Figure \ref{project-19-git-commit}, and then type in
your username and email address on GitHub in the popup ``Identify Yourself''
message window.

\begin{figure}[t]
\centering
\includegraphics[scale=0.3]{project-21-git-commit-message}
\caption{Viewing and confirming commit message\label{project-21-git-commit-message}}
\end{figure}

\begin{figure}[t]
\centering
\includegraphics[scale=0.3]{project-22-git-commit-done}
\caption{Committed project\label{project-22-git-commit-done}}
\end{figure}

\item In the ``Commit Changes'' window, you are allowed to choose the files you
want to commit (and also automatically add to the index). As you can see in
Figure \ref{project-21-git-commit-message}, during creating the empty Maven
project, several Eclipse and Maven related configuration files are generated.
Moreover, type in a commit message to describe what changes you've made and
leave your name as well as your email address (which is a convention) in the
committer field. Finally, by clicking \textbf{Commit}, you've done with your
first git-commit. You can see in Figure \ref{project-22-git-commit-done}, the
``greater than'' symbol disappears, and the question marks on the committed
files become a repository icon, which means the files are in the latest version.
You could also find your git-commit helps the code merge into the new master
branch from a NO-HEAD branch.

If you are using SVN, then you've done with sychronizing your local workspace
with the remote repository once you execute a commit command, but remember the
feature of Git? Your project repository is distributed, which means your
previous git-commit conceptually affects all your project repositories, but in
fact you should make it happen with an additional \emph{push}. A nice picture at
\url{http://gitready.com/beginner/2009/01/21/pushing-and-pulling.html} may help
you better understand what is actually happening when you perform git commit,
add, push, fetch, pull, etc.

\begin{figure}[t]
\centering
\includegraphics[scale=0.3]{project-23-git-push}
\caption{Performing a git push\label{project-23-git-push}}
\end{figure}

\item Right-click on the project name, and select \textbf{Team} $\rightarrow$
\textbf{Push to Upstream}. See Figure \ref{project-23-git-push}.

\begin{figure}[t]
\hspace{-2em}
\begin{minipage}{0.5\textwidth}
\centering
\includegraphics[scale=0.3]{project-24-git-push-progress}
\caption{Viewing the git push progress\label{project-24-git-push-progress}}
\end{minipage}
\hfill
\begin{minipage}{0.5\textwidth}
\centering
\includegraphics[scale=0.3]{project-25-git-push-result}
\caption{Viewing the git push result\label{project-25-git-push-result}}
\end{minipage}
\hspace{-1em}
\end{figure}

\item Now you can see a progress indication window pops up (see Figure
\ref{project-24-git-push-progress}), which says \textbf{Pushing to remote
repositories}.

\item Finally, you could see the push results in the ``Push Results'' window.
Click \textbf{OK} to close the window and get back to your workspace.

\end{enumerate}

In your next homework, you will need to use other Git commands within Eclipse.



\chapter{Release to Maven}

In the last task, you will need to submit your Java code by performing a release of your code. But remember that Maven release plug-in will check if all the changes you have made have been checked into the remote repository (i.e. GitHub in our case). So, let's perform a git-commit and a git-push.

\begin{enumerate}

\begin{figure}
\hspace{-1em}
\begin{minipage}{0.5\textwidth}
\centering
\includegraphics[scale=0.3]{simple-code-05-commit}
\caption{Performing a git-commit/push before preparing a release\label{simple-code-05-commit}}
\end{minipage}
\hfill
\begin{minipage}{0.5\textwidth}
\centering
\includegraphics[scale=0.3]{submit-01-preference}
\caption{Starting to add an external Maven executable\label{submit-01-preference}}
\end{minipage}
\hspace{-1em}
\end{figure}

\item Similar to what you did earlier, you execute git-commit and git-push to the project, and you could see the ``greater than'' symbol disappears and a ``master'' label is attached to project path, which means you are successful with your git-commit and git-push.

\begin{figure}
\hspace{-2em}
\begin{minipage}{0.5\textwidth}
\centering
\includegraphics[scale=0.3]{submit-02-add-maven}
\caption{Adding another Maven executable\label{submit-02-add-maven}}
\end{minipage}
\hfill
\begin{minipage}{0.5\textwidth}
\centering
\includegraphics[scale=0.3]{submit-03-add-maven-done}
\caption{Viewing the added external Maven\label{submit-03-add-maven-done}}
\end{minipage}
\hspace{-2em}
\end{figure}

\item Sometimes, the embedded Maven runtime from m2e cannot be succesfully executed to perform a release goal. Therefore, we should add the externally installed Maven runtime into the Eclipse. Click \textbf{Window} (or \textbf{Edit}) $\rightarrow$ \textbf{Preferences} (see Figure \ref{submit-03-add-maven-done}).
\item Select \textbf{Maven} $\rightarrow$ \textbf{Installations}. You will see the ``Embedded'' runtime (as shown in Figure \ref{submit-02-add-maven}). Click \textbf{Add\ldots} to locate the installation path of your Maven runtime.
\item Then, you could see an ``External'' runtime in the installation lists. See Figure \ref{submit-03-add-maven-done}.

\begin{figure}
\centering
\includegraphics[scale=0.3]{submit-04-run-config}
\caption{Getting ready for a release\label{submit-04-run-config}}
\end{figure}

\item Now you can execute a Maven goal within Eclipse (of course, you can also do that outside Eclipse from command line). Click the down-arrow next to the \textbf{Run} button, and select \textbf{Run Configurations\ldots}. See Figure \ref{submit-04-run-config}.

\begin{figure}
\centering
\includegraphics[scale=0.3]{submit-05-run-config-done}
\caption{Configuring Maven goal\label{submit-05-run-config-done}}
\end{figure}

\item In the ``Run Configurations'' window, double-click \textbf{Maven Build} to create a new Maven goal. See Figure \ref{submit-05-run-config-done}. Rename your run configuration name as ``release'' (optional), and click \textbf{Browse Workspace\ldots} to select your project, and type in your goals as follows:

\begin{verbatim}
release:prepare release:perform
\end{verbatim}

This actually defines two goals ``release:prepare'' and ``release:perform''. As you will probably encounter tons of errors during this step, we should review some details of the Maven release.

The Maven guide\footnote{\url{https://maven.apache.org/guides/mini/guide-releasing.html}} tells us what is happening behind these two goals:

\begin{quote}
The release:prepare goal will:

\begin{enumerate}
\item Verify that there are no uncommitted changes in the workspace.
\item Prompt the user for the desired tag, release and development version names.
\item Modify and commit release information into the pom.xml file.
\item Tag the entire project source tree with the new tag name.
\end{enumerate}

The release:perform goal will:

\begin{enumerate}
\item Extract file revisions versioned under the new tag name.
\item Execute the maven build lifecycle on the extracted instance of the project.
\item Deploy the versioned artifacts to appropriate local and remote repositories.
\end{enumerate}
\end{quote}

If the goals are not executed sucessfully, a relatively useful message will be printed out to console to help you discover where the problem is.

\begin{figure}
\centering
\includegraphics[scale=0.3]{submit-06-release}
\caption{A successful release!\label{submit-06-release}}
\end{figure}

\item Finally after you fixed all the problems (if any), you could see the very pleasant ``BUILD SUCCESS'' message in the console (as shown in Figure \ref{submit-06-release}), which means you've done with your Homework 0! Congratulations!

\begin{qa}
\item[Q1] What if I find some bugs after it has been released? How can I resubmit my code?
\item[A1] You can look at the ``Overview'' tab in the Maven POM Editor for your pom.xml file. The version should be ``0.0.2-SNAPSHOT'', which indicates a previous version has been generated. Now, you could redo this task (git-commit, git-push, run Maven goals) to release the version 0.0.2. We will evaluate your code based on the latest release.
\end{qa}

\end{enumerate}


\section{Eclipse (Git, Maven plug-ins integrated)}

If you have an Eclipse IDE for Java Developers with version $\ge$ 3.7, you could probably skip this task. But if you are stuck in a situation where you were told Eclipse is missing a plug-in, you might want to return to this section. If you have other packages (Eclipse Classic or Eclipse for Java EE Developers), please do not skip this task.

\begin{enumerate}
\item Download Eclipse IDE for Java Developers 4.2 at \url{http://www.eclipse.org/downloads/packages/eclipse-ide-java-developers/junor}.

\begin{framed}
\begin{itemize}
\item[Q1] Can I use an older version of Eclipse?
\item[A1] You can try to use an older version, but some plug-ins (e.g., m2e) might complain about the Eclipse version if it is older than 3.7.
\item[Q2] Can I use other Eclipse packages?
\item[A2] Eclipse IDE for Java Developers includes almost all the Eclipse components (jdt, EGit, m2e, and so on) we need for this course. You could also work with other packages, e.g., Eclipse IDE for Jave EE Developers or Eclipse Classics, and as they don't come with such plugins by default, you have to install these plug-ins all by yourself.
\end{itemize}
\end{framed}

\item Install Eclipse by simply uncompressing the downloaded package.

\begin{figure}[t]
\centering
\includegraphics[scale=0.3]{eclipse-01-login}
\caption{Eclipse choose workspace\label{eclipse-01-login}}
\end{figure}

\item Use the default workspace path or create your own workspace, as shown in Figure \ref{eclipse-01-login}. And finally, we could see the Eclipse Welcome view at the end of workspace initialization. See Figure \ref{eclipse-02-welcome}.

\begin{figure}[t]
\centering
\includegraphics[scale=0.3]{eclipse-02-welcome}
\caption{Eclipse Welcome view\label{eclipse-02-welcome}}
\end{figure}

\end{enumerate}


That's the start of your developement. From now on, everthing will become less
platform specific, and we will show you how to configure the workspace, create
your Maven project and release it in the rest of the homework.


\section{A Little Configuration}

\subsection{Letting m2e know your password}

\begin{enumerate}
\item Click \textbf{Edit} (or \textbf{Window} depending on your OS) $\rightarrow$ \textbf{Preferences}, and choose \textbf{Maven} $\rightarrow$ \textbf{User Settings}, and find the default Maven setting path for your system. See Figure \ref{eclipse-03-maven-setting}.

\item Create the \verb|settings.xml| file at the given directory, and copy the text in Listing \ref{settings} into the file, which will store your ID and passwords. Remember to replace \verb|ID| and \verb|PASSWORD| with your personal Maven project repository account we provide you, and also don't upload this file to any remote repository or share it with others.

\item Go back to \textbf{Edit} (or \textbf{Window}) $\rightarrow$ \textbf{Preferences}, and choose \textbf{Maven} $\rightarrow$ \textbf{User Settings} again, you will see the plugin could find the setting file you specified (see Figure \ref{eclipse-04-maven-setting-back}).
% and click \textbf{Update settings}. You will be able to see the repository is now being refreshed. 

\begin{figure}[t]
\hspace{-3em}
\begin{minipage}{0.5\textwidth}
\centering
\includegraphics[scale=0.3]{eclipse-03-maven-setting}
\caption{Eclipse maven user profile setting\label{eclipse-03-maven-setting}}
\end{minipage}
\hfill
\begin{minipage}{0.5\textwidth}
\centering
\includegraphics[scale=0.3]{eclipse-04-maven-setting-back}
\caption{Eclipse maven user profile setting\label{eclipse-04-maven-setting-back}}
\end{minipage}
\hspace{-3em}
\end{figure}

\lstinputlisting[language=XML,float,linewidth=1.1\textwidth,caption=Configuring settings.xml,label=settings]{../lst/settings.xml}

\end{enumerate}

\subsection{Importing Apache UIMA code style template}

To development a software as a team, members should always adopt the same code conventions to improve the readability and maintainability of the project. We suggest you to view the \emph{Code Conventions for the Java Programming Language} at \url{http://www.oracle.com/technetwork/java/codeconv-138413.html}, which was published from Oracle. For our course homeworks, you are required to adopt a set of more specific coding conventions from Apache UIMA project. Details can be found at \url{http://uima.apache.org/codeConventions.html}. At the bottom of the page, you could find a link to download the Eclipse code style template\footnote{\url{http://uima.apache.org/downloads/ApacheUima_EclipseCodeStylePrefs.xml}}.

\begin{enumerate}
\item Download the template and save it in your local filesystem.
\item Click \textbf{Window} $\rightarrow$ \textbf{Preferences}, then go to \textbf{Java} $\rightarrow$ \textbf{Code Style} $\rightarrow$ \textbf{Formatter}, and click \textbf{Import\ldots}. 
\end{enumerate}

Remember before you finish editing a Java file, press \textbf{Ctrl+Shift+F} to perform an automatic code formation.

Another optional but useful tool for you to check your code style is the Eclipse Checkstyle plug-in. You can learn how to download and install the plug-in at \url{http://eclipse-cs.sourceforge.net/}.




\chapter{Installing and Configuring Softwares}

In this task, you are required to install the tools you've already been familiar
with (at least heard about).

\textbf{Important notes:} This might be the hardest part of your Homework 0,
since installation of the same software on different platforms differs a lot,
which means we could not show you detailed steps for each platform. Instead, we
will provide you with links for the installation instructions for you to follow,
and we have created three forums for you to discuss Windows/Linux/Mac platform
related issues. We will try our best to help you solve whatever problems you
might have, but we also encourage you, the experts in any of these platforms, to
work with us to answer the questions from your classmates. You can find the
discussion boards at \url{http://blackboard.andrew.cmu.edu}, and select the
course, then the \textbf{Tools} and \textbf{Discussion Board}.


\section{Installing JDK}

If you have the latest JDK 6 installed\footnote{You should check out the latest
version at
\url{http://www.oracle.com/technetwork/java/javase/downloads/index.html} as the
version number grows really fast}, you could skip this task.

We assume you have experience in Java programming, but we still need to clarify
the Java environment for the course. If you don't have any Java experience,
probably you need to look for a Java textbook. It might also be fine if you
think you have tons of experience in programming in C++/C\# and you feel
confident to learn Java by just reading others' codes and guessing their
meanings. It's up to you!

\begin{enumerate}

\item Visit
\url{http://www.oracle.com/technetwork/java/javase/downloads/index.html}, and
choose the platform you are using to download JDK 6 SE 35\footnote{By the date
of August 31, 2012}.

\begin{qa}

\item[Q1] Can I just install JRE instead of JDK?

\item[A1] No.

\item[Q2] Can I install OpenJDK instead of SunJDK (or OracleJDK)?

\item[A2] Sure, you can. But be aware that sometimes only binary files (aka JRE)
are installed under a folder named \texttt{openjdk-\emph{version}}, rather than
\texttt{openjre-\emph{version}}, which is a bit confusing.

\item[Q3] Can I install JDK 7, 5 or older versions?

\item[A3] You are not recommended to install JDK 7, since you have to modify the
Maven pom file to compile your project, and the cluster that we will run and
test your components does not have JDK 7 set up yet. But it would be fine
(theoretically) if you have just JDK 5 installed, but it is still not
recommended. Versions older than 5 should be completely avoided.

\item[Q4] Can I use an earlier version of JDK 6 (e.g., 6u4)?

\item[A4] It may not put you in a trouble most of time, but in some rare cases,
we did find an exception was thrown due to a bug not in our code but in the
runtime environment. Therefore, we recommend you to upgrade your JDK 6 to the
latest version.

\end{qa}

\item Install JDK from the executable file if available, and set PATH manually
(if you are using a Windows machine). The Java installation page (at
\url{http://www.oracle.com/technetwork/java/javase/index-137561.html}) might be
useful to you.

\end{enumerate}


\section{Commiting and Pushing Your Maven Project Back to Git}

\begin{figure}[t]
\centering
\includegraphics[scale=0.3]{project-19-git-commit}
\caption{Performing a git-commit\label{project-19-git-commit}}
\end{figure}

You should see a ``greater than'' symbol between the project icon and your
project name (see Figure \ref{project-19-git-commit}, which means you have made
some changes to the project so that there exist some differences between your
current workspace version and the branch head. Question marks on the project
element icon represent the corresponding elements are not indexed yet. You might
be wondering what changes you've made because you thought you haven't written
any code. Actually, you have created a Maven project, which process will
generate the \verb|pom.xml| file, and modified the Eclipse configuration files.
So let's perform our first git-commit.

\begin{enumerate}

\item Right-click on the project name, and select \textbf{Team} $\rightarrow$
\textbf{Commit\ldots}. See Figure \ref{project-19-git-commit}, and then type in
your username and email address on GitHub in the popup ``Identify Yourself''
message window.

\begin{figure}[t]
\centering
\includegraphics[scale=0.3]{project-21-git-commit-message}
\caption{Viewing and confirming commit message\label{project-21-git-commit-message}}
\end{figure}

\begin{figure}[t]
\centering
\includegraphics[scale=0.3]{project-22-git-commit-done}
\caption{Committed project\label{project-22-git-commit-done}}
\end{figure}

\item In the ``Commit Changes'' window, you are allowed to choose the files you
want to commit (and also automatically add to the index). As you can see in
Figure \ref{project-21-git-commit-message}, during creating the empty Maven
project, several Eclipse and Maven related configuration files are generated.
Moreover, type in a commit message to describe what changes you've made and
leave your name as well as your email address (which is a convention) in the
committer field. Finally, by clicking \textbf{Commit}, you've done with your
first git-commit. You can see in Figure \ref{project-22-git-commit-done}, the
``greater than'' symbol disappears, and the question marks on the committed
files become a repository icon, which means the files are in the latest version.
You could also find your git-commit helps the code merge into the new master
branch from a NO-HEAD branch.

If you are using SVN, then you've done with sychronizing your local workspace
with the remote repository once you execute a commit command, but remember the
feature of Git? Your project repository is distributed, which means your
previous git-commit conceptually affects all your project repositories, but in
fact you should make it happen with an additional \emph{push}. A nice picture at
\url{http://gitready.com/beginner/2009/01/21/pushing-and-pulling.html} may help
you better understand what is actually happening when you perform git commit,
add, push, fetch, pull, etc.

\begin{figure}[t]
\centering
\includegraphics[scale=0.3]{project-23-git-push}
\caption{Performing a git push\label{project-23-git-push}}
\end{figure}

\item Right-click on the project name, and select \textbf{Team} $\rightarrow$
\textbf{Push to Upstream}. See Figure \ref{project-23-git-push}.

\begin{figure}[t]
\hspace{-2em}
\begin{minipage}{0.5\textwidth}
\centering
\includegraphics[scale=0.3]{project-24-git-push-progress}
\caption{Viewing the git push progress\label{project-24-git-push-progress}}
\end{minipage}
\hfill
\begin{minipage}{0.5\textwidth}
\centering
\includegraphics[scale=0.3]{project-25-git-push-result}
\caption{Viewing the git push result\label{project-25-git-push-result}}
\end{minipage}
\hspace{-1em}
\end{figure}

\item Now you can see a progress indication window pops up (see Figure
\ref{project-24-git-push-progress}), which says \textbf{Pushing to remote
repositories}.

\item Finally, you could see the push results in the ``Push Results'' window.
Click \textbf{OK} to close the window and get back to your workspace.

\end{enumerate}

In your next homework, you will need to use other Git commands within Eclipse.



\chapter{Release to Maven}

In the last task, you will need to submit your Java code by performing a release of your code. But remember that Maven release plug-in will check if all the changes you have made have been checked into the remote repository (i.e. GitHub in our case). So, let's perform a git-commit and a git-push.

\begin{enumerate}

\begin{figure}
\hspace{-1em}
\begin{minipage}{0.5\textwidth}
\centering
\includegraphics[scale=0.3]{simple-code-05-commit}
\caption{Performing a git-commit/push before preparing a release\label{simple-code-05-commit}}
\end{minipage}
\hfill
\begin{minipage}{0.5\textwidth}
\centering
\includegraphics[scale=0.3]{submit-01-preference}
\caption{Starting to add an external Maven executable\label{submit-01-preference}}
\end{minipage}
\hspace{-1em}
\end{figure}

\item Similar to what you did earlier, you execute git-commit and git-push to the project, and you could see the ``greater than'' symbol disappears and a ``master'' label is attached to project path, which means you are successful with your git-commit and git-push.

\begin{figure}
\hspace{-2em}
\begin{minipage}{0.5\textwidth}
\centering
\includegraphics[scale=0.3]{submit-02-add-maven}
\caption{Adding another Maven executable\label{submit-02-add-maven}}
\end{minipage}
\hfill
\begin{minipage}{0.5\textwidth}
\centering
\includegraphics[scale=0.3]{submit-03-add-maven-done}
\caption{Viewing the added external Maven\label{submit-03-add-maven-done}}
\end{minipage}
\hspace{-2em}
\end{figure}

\item Sometimes, the embedded Maven runtime from m2e cannot be succesfully executed to perform a release goal. Therefore, we should add the externally installed Maven runtime into the Eclipse. Click \textbf{Window} (or \textbf{Edit}) $\rightarrow$ \textbf{Preferences} (see Figure \ref{submit-03-add-maven-done}).
\item Select \textbf{Maven} $\rightarrow$ \textbf{Installations}. You will see the ``Embedded'' runtime (as shown in Figure \ref{submit-02-add-maven}). Click \textbf{Add\ldots} to locate the installation path of your Maven runtime.
\item Then, you could see an ``External'' runtime in the installation lists. See Figure \ref{submit-03-add-maven-done}.

\begin{figure}
\centering
\includegraphics[scale=0.3]{submit-04-run-config}
\caption{Getting ready for a release\label{submit-04-run-config}}
\end{figure}

\item Now you can execute a Maven goal within Eclipse (of course, you can also do that outside Eclipse from command line). Click the down-arrow next to the \textbf{Run} button, and select \textbf{Run Configurations\ldots}. See Figure \ref{submit-04-run-config}.

\begin{figure}
\centering
\includegraphics[scale=0.3]{submit-05-run-config-done}
\caption{Configuring Maven goal\label{submit-05-run-config-done}}
\end{figure}

\item In the ``Run Configurations'' window, double-click \textbf{Maven Build} to create a new Maven goal. See Figure \ref{submit-05-run-config-done}. Rename your run configuration name as ``release'' (optional), and click \textbf{Browse Workspace\ldots} to select your project, and type in your goals as follows:

\begin{verbatim}
release:prepare release:perform
\end{verbatim}

This actually defines two goals ``release:prepare'' and ``release:perform''. As you will probably encounter tons of errors during this step, we should review some details of the Maven release.

The Maven guide\footnote{\url{https://maven.apache.org/guides/mini/guide-releasing.html}} tells us what is happening behind these two goals:

\begin{quote}
The release:prepare goal will:

\begin{enumerate}
\item Verify that there are no uncommitted changes in the workspace.
\item Prompt the user for the desired tag, release and development version names.
\item Modify and commit release information into the pom.xml file.
\item Tag the entire project source tree with the new tag name.
\end{enumerate}

The release:perform goal will:

\begin{enumerate}
\item Extract file revisions versioned under the new tag name.
\item Execute the maven build lifecycle on the extracted instance of the project.
\item Deploy the versioned artifacts to appropriate local and remote repositories.
\end{enumerate}
\end{quote}

If the goals are not executed sucessfully, a relatively useful message will be printed out to console to help you discover where the problem is.

\begin{figure}
\centering
\includegraphics[scale=0.3]{submit-06-release}
\caption{A successful release!\label{submit-06-release}}
\end{figure}

\item Finally after you fixed all the problems (if any), you could see the very pleasant ``BUILD SUCCESS'' message in the console (as shown in Figure \ref{submit-06-release}), which means you've done with your Homework 0! Congratulations!

\begin{qa}
\item[Q1] What if I find some bugs after it has been released? How can I resubmit my code?
\item[A1] You can look at the ``Overview'' tab in the Maven POM Editor for your pom.xml file. The version should be ``0.0.2-SNAPSHOT'', which indicates a previous version has been generated. Now, you could redo this task (git-commit, git-push, run Maven goals) to release the version 0.0.2. We will evaluate your code based on the latest release.
\end{qa}

\end{enumerate}


\section{Eclipse (Git, Maven plug-ins integrated)}

If you have an Eclipse IDE for Java Developers with version $\ge$ 3.7, you could probably skip this task. But if you are stuck in a situation where you were told Eclipse is missing a plug-in, you might want to return to this section. If you have other packages (Eclipse Classic or Eclipse for Java EE Developers), please do not skip this task.

\begin{enumerate}
\item Download Eclipse IDE for Java Developers 4.2 at \url{http://www.eclipse.org/downloads/packages/eclipse-ide-java-developers/junor}.

\begin{framed}
\begin{itemize}
\item[Q1] Can I use an older version of Eclipse?
\item[A1] You can try to use an older version, but some plug-ins (e.g., m2e) might complain about the Eclipse version if it is older than 3.7.
\item[Q2] Can I use other Eclipse packages?
\item[A2] Eclipse IDE for Java Developers includes almost all the Eclipse components (jdt, EGit, m2e, and so on) we need for this course. You could also work with other packages, e.g., Eclipse IDE for Jave EE Developers or Eclipse Classics, and as they don't come with such plugins by default, you have to install these plug-ins all by yourself.
\end{itemize}
\end{framed}

\item Install Eclipse by simply uncompressing the downloaded package.

\begin{figure}[t]
\centering
\includegraphics[scale=0.3]{eclipse-01-login}
\caption{Eclipse choose workspace\label{eclipse-01-login}}
\end{figure}

\item Use the default workspace path or create your own workspace, as shown in Figure \ref{eclipse-01-login}. And finally, we could see the Eclipse Welcome view at the end of workspace initialization. See Figure \ref{eclipse-02-welcome}.

\begin{figure}[t]
\centering
\includegraphics[scale=0.3]{eclipse-02-welcome}
\caption{Eclipse Welcome view\label{eclipse-02-welcome}}
\end{figure}

\end{enumerate}


That's the start of your developement. From now on, everthing will become less
platform specific, and we will show you how to configure the workspace, create
your Maven project and release it in the rest of the homework.


\section{A Little Configuration}

\subsection{Letting m2e know your password}

\begin{enumerate}
\item Click \textbf{Edit} (or \textbf{Window} depending on your OS) $\rightarrow$ \textbf{Preferences}, and choose \textbf{Maven} $\rightarrow$ \textbf{User Settings}, and find the default Maven setting path for your system. See Figure \ref{eclipse-03-maven-setting}.

\item Create the \verb|settings.xml| file at the given directory, and copy the text in Listing \ref{settings} into the file, which will store your ID and passwords. Remember to replace \verb|ID| and \verb|PASSWORD| with your personal Maven project repository account we provide you, and also don't upload this file to any remote repository or share it with others.

\item Go back to \textbf{Edit} (or \textbf{Window}) $\rightarrow$ \textbf{Preferences}, and choose \textbf{Maven} $\rightarrow$ \textbf{User Settings} again, you will see the plugin could find the setting file you specified (see Figure \ref{eclipse-04-maven-setting-back}).
% and click \textbf{Update settings}. You will be able to see the repository is now being refreshed. 

\begin{figure}[t]
\hspace{-3em}
\begin{minipage}{0.5\textwidth}
\centering
\includegraphics[scale=0.3]{eclipse-03-maven-setting}
\caption{Eclipse maven user profile setting\label{eclipse-03-maven-setting}}
\end{minipage}
\hfill
\begin{minipage}{0.5\textwidth}
\centering
\includegraphics[scale=0.3]{eclipse-04-maven-setting-back}
\caption{Eclipse maven user profile setting\label{eclipse-04-maven-setting-back}}
\end{minipage}
\hspace{-3em}
\end{figure}

\lstinputlisting[language=XML,float,linewidth=1.1\textwidth,caption=Configuring settings.xml,label=settings]{../lst/settings.xml}

\end{enumerate}

\subsection{Importing Apache UIMA code style template}

To development a software as a team, members should always adopt the same code conventions to improve the readability and maintainability of the project. We suggest you to view the \emph{Code Conventions for the Java Programming Language} at \url{http://www.oracle.com/technetwork/java/codeconv-138413.html}, which was published from Oracle. For our course homeworks, you are required to adopt a set of more specific coding conventions from Apache UIMA project. Details can be found at \url{http://uima.apache.org/codeConventions.html}. At the bottom of the page, you could find a link to download the Eclipse code style template\footnote{\url{http://uima.apache.org/downloads/ApacheUima_EclipseCodeStylePrefs.xml}}.

\begin{enumerate}
\item Download the template and save it in your local filesystem.
\item Click \textbf{Window} $\rightarrow$ \textbf{Preferences}, then go to \textbf{Java} $\rightarrow$ \textbf{Code Style} $\rightarrow$ \textbf{Formatter}, and click \textbf{Import\ldots}. 
\end{enumerate}

Remember before you finish editing a Java file, press \textbf{Ctrl+Shift+F} to perform an automatic code formation.

Another optional but useful tool for you to check your code style is the Eclipse Checkstyle plug-in. You can learn how to download and install the plug-in at \url{http://eclipse-cs.sourceforge.net/}.



\end{document}
